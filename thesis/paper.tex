\documentclass[notitlepage,twoside,a4paper]{scrreprt}

\usepackage{amssymb,qtree,natbib,bbm,aleks,linguex}

\usepackage{stmaryrd,amsthm,amsmath}

%% note that this file depends on being processed with XeTeX
\usepackage{fontspec}
\usepackage{xltxtra,xunicode}

\usepackage{listings}

\usepackage{mathpazo}
% font specifications (don't overuse ;-)
\defaultfontfeatures{Mapping=tex-text} % converts LaTeX specials (``quotes'' --- dashes etc.) to unicode
\setromanfont [Ligatures={Common,Rare}, BoldFont={Fontin Bold}, ItalicFont={Fontin Italic}]{Fontin Regular}
\setsansfont [Ligatures={Common}, BoldFont={Fontin Sans Bold}, ItalicFont={Fontin Sans Italic}]{Fontin Sans}
%\setmonofont [Ligatures={NoCommon,NoRequired,NoRare}] {Monaco}

\usepackage{tuebingerspruechle}

\newcommand{\abbr}{\textsf} % how to render abbreviations like IL or Ty2
\newcommand{\stress}{\textbf} % put stress somewhere (as opposed to \emph)
\newcommand{\term}[1]{\textsf{\textbf{#1}}} % introduce a term
\newcommand{\code}[1]{\texttt{#1}} % layout of in-line code snippets
\newcommand{\pn}{\textsf} % how to render proper names like Shrdlu
\newcommand{\url}[1]{\code{http://#1}} % how urls appear
\newcommand{\example}[1]{`\textit{#1}'} % in-text examples

% commonly used phrases
\newcommand{\Disc}{\ensuremath{\mathcal{D}}} % the discourse
\newcommand{\wh}{\textsc{wh}}

%quotations
\newenvironment{quotes}{\begin{quote}\sf}{\rm\end{quote}}
\newcommand{\quotesf}[1]{\quote{\textsf{#1}}}

%names
\newcommand{\curt}{\pn{Curt}}
\newcommand{\acurt}{\pn{Advertent Curt}}
\newcommand{\prol}{\pn{Prolog}}
\pagestyle{headings}

%theorem environments
\newtheorem{thm}{Theorem}[section]
\newtheorem{cor}[thm]{Corollary}
\newtheorem{lem}[thm]{Lemma}

\theoremstyle{remark}
\newtheorem{rem}[thm]{Remark}
\theoremstyle{remark}
\newtheorem*{termin}{Terminology}

\theoremstyle{definition}
\newtheorem{definition}[thm]{Definition}

\theoremstyle{definition}
\newtheorem{postulate}[thm]{Postulate}

\renewcommand{\labelenumi}{(\roman{enumi})}

\bibliographystyle{plainnat}
\bibpunct{(}{)}{;}{a}{,}{,}

\author{Aleksandar Lyubenov Dimitrov\\
$\langle$\href{mailto:aleks.dimitrov@googlemail.com}{aleks.dimitrov@googlemail.com}$\rangle$}
\title{From Questions to Queries – Enhancing the Curt System by a Theory and Implementation of Embedded WH-Questions}

\usepackage{fontspec}
\usepackage[xetex,colorlinks]{hyperref}
\begin{document}
\lstset{language=Prolog,numbers=left,numberstyle=\tiny,numbersep=5pt,stepnumber=2,basicstyle=\ttfamily,identifierstyle=\textbf,keywordstyle=\underbar}
\renewcommand{\thepage}{\roman{page}}
\begin{titlepage}
  \maketitle
  \thispagestyle{empty}
  \begin{center}
    \vskip 4em
    Thesis submitted for the Degree of Bachelor of Arts\\

    \paragraph{Supervisor:} Dr. Frank Richter
    \vskip 6em
    Seminar für Sprachwissenschaft\\
    Eberhard-Karls-Universität\\
    Tübingen
  \end{center}
\end{titlepage}
\mbox{}
\newpage
\tableofcontents
\newpage
\SpruechleAufsagen{Aleksandar L. Dimitrov}
\thispagestyle{plain}
\newpage
\renewcommand{\thepage}{\arabic{page}}
\setcounter{page}{1}
\chapter{Introduction}

As access to vast amounts of information becomes increasingly ubiquitous, and
the information accessible to humans grows more diverse every day, users
struggle to find what they need among the vast amounts of data available.
Current search technologies aim to address this problem by providing quick and
easy access to a constantly changing pool of knowledge, yet they are hardly
beyond the stage of looking for exact text matches. Machine processing still
fails to \emph{understand} the information it has to deal with and often sees data as
nothing more than a set of strings. But what if a user's demands do not ask for a
piece of text, but for an abstract concept, or a relation? Knowing that natural
language is most likely infinite, such relations are virtually endless in
number, and  one can only conclude that trying to enumerate all relations
between the bits of  human knowledge in gargantuan databases is bound to be a
never-ending task.

Growing efforts have thus been made to provide a technology that is (going to
be) able to extract logical formulae from sentences of ordinary language and
establish relations between objects of their logical representation. In
addition, providing a means to compute with said representations will eventually
enable the machine-processing of an abstract concept of meaning. In doing so,
the ultimate goal may be to enable a man-made system to \emph{understand} what
humans communicate.  These systems could then be queried and, given a capability
of performing logical inference on their domains of knowledge, they could
compute what an optimal answer to any given query might be -- perhaps even
by inventing an answer to a question never posed before.

Exploration of this  interesting field of linguistics has shown that
the notion of \emph{understanding} is far more than a linguistic concern, and
involves at least findings in the domain of logics, (cognitive) psychology and
probably even philosophy.

Development of exhaustive systems in this area appears to be a daunting task of
quite some scope. Therefore most approaches have been focused on providing
coverage of a small fragment of natural language, often with respect to a
particular domain. Many such systems exist and they prove to be increasingly
successful. Some of them will be discussed here, and presenting an extension to
one system in particular is the main matter of this Bachelor's thesis.

\subsection{Overview}

The present work aims to provide the \curt{} system of
\cite{blackburnbos:cl1} with the mechanics of dealing with \term{embedded
questions}, which enables it to reason not only about its own knowledge, but
also about the knowledge of individuals in its target domain.
For this, the system will have to enhanced with the capability of handling more
than one partition of its logical domain, viz. the ability to handle several
\term{possible} or \term{accessible worlds}.

As the scope of a Bachelor's thesis is limited by its nature, not all of the
theory and reasoning can be explained and background knowledge in some rather
divergent domains is assumed. In particular, a basic familiarity with
traditional \term{natural language semantics} and \term{extensional} as well as
\term{intensional logic} or \term{modal} frameworks will be assumed.
Furthermore, a basic acquaintance with theories of the logic of questions will be helpful,
but not strictly necessary in understanding some of the reasoning in section
\ref{sec:theory}.  Some knowledge of the \prol{} programming language and a good
overview of the architecture of the \curt{} system is imperative in order to
understand the implementation itself.

Section \ref{sec:concepts}  gives a conceptual overview of the technologies to
date and the particular software, \curt, that will serve as a basis for
extension.  The next section concerns itself with a more detailed analysis of
the theoretic assumptions of this implementation and justifies some choices and
restrictions that had to be made due to the limited scope of this analysis.
While the section \ref{sec:implementation} displays and documents the actual
implementation on top of the \curt{} system, the last section discusses the
results of the previous sections and also give a prospect on possible further work on the
subject. Finally, the appendix holds the sources code and some example
computations and models that show the basic functionality of \curt.

\subsection{Typographic Conventions}

The following typographic conventions are used throughout the document to
increase readability:

\begin{itemize}
  \item Text in a bold face sans serif font marks new terminology and important
  concepts: \term{embedded interrogatives}, \term{backtracking}
  \item Proper names, names of technologies, and abbreviations are typeset in a
  regular sans serif font: \acurt, \prol, \pn{Ty2}
  \item Code is written in a monospace font, resembling the look of a font class most
  programmers are used to:
  \begin{lstlisting}
    lam(P, lam(Q, all(X, imp(app(P,X),app(Q,X)))))
  \end{lstlisting}
  \item Logical formulae are typeset in a different serif font from the rest of
  the text:\\
    Assume $k = (\lambda x.\neg xx)$. Then $ kk = (\lambda x .\neg xx)k = \neg kk$
  \item Quotations  are typeset on a separate paragraph, using a sans serif font:
  \quotesf{This answers the question!}
\end{itemize}


\section{Concepts}\label{sec:concepts}

Computational Semantics is a discipline that has  lately gained a wider focus in
literature due to recent progress in different areas.  Current development in
computer hardware has made it possible to focus on theoretical soundness,
important abstract designs, and concepts of logic and the theory of meaning,
rather than being able to simply implement a system that is able to do basic
computations in a reasonable amount of time. On the other hand, research in both
computer science and linguistics has fueled new ideas and intriguing hypotheses
emerge from lively discussions in their respective fields on a very frequent
basis.

This section delineates the ground the software presented in this thesis is
based on while putting it into this rapidly evolving context. \ref{sec:curt}
introduces the base system of the implementation, \curt{}, and
\ref{sec:comparison} relates it to other similar systems in existence.
\ref{sec:extension} then goes on to explain where the \curt{} system lacks in
features and thus exhibits potential for extension.

\subsection{The Curt System} \label{sec:curt}

\curt, a system written in \prol{} that is able to interpret basic utterances
of English and store them in expressions of first order logic, is
presented in \cite[chapter 6]{blackburnbos:cl1}. While relatively limited when
compared to other systems presented in \ref{sec:comparison} in terms of lexical
and grammatical coverage, it provides a sound basis for extended work due to
its coherent design and relative simplicity.

Besides parsing and translating utterances into logic, \curt{} is able to perform
basic checks for \emph{consistency} and \emph{informativity} on the resulting
formulae -- within the bounds of the tools it uses\footnote{\curt{} makes use
of (among others) the \pn{Otter} theorem prover and the \pn{mace2} model
builder. During the process of the implementation of section
\ref{sec:implementation}, it was ported to also make use of the \pn{Prover9}
theorem prover.}.
It can thus detect incoherent discourse and would reject a sequence of
sentences such as \emph{``Mary likes every man. Peter is a man. Mary does not
like Peter''}, claiming it found a contradiction in its input. The discourse is
then said to be \term{inconsistent} with the new input.

For every input
sentence, \curt{} will also try to infer whether the current discourse would
entail the information it contains and would thus be \term{uninformative} with
respect to previous utterances – effectively allowing it to
reduce the size of its internal representation of the discourse by avoiding to
add clauses to its database that would not contribute to the overall informational value
already present in the program.

The final chapter of \cite{blackburnbos:cl1} introduces the implementation
incrementally, presenting new versions of \curt{} that add to the software's
feature set one at a time. Support for inferencing via external theorem provers
and model builders is added right after the first, rather limited implementations.
Finally an ontology and a straight-forward way of parsing and answering direct
questions for noun phrases conclude its development. The theory and
implementation of this paper are meant as a replacement and enhancement for the
latter feature. 
%Their introduction will also happen in an incremental manner and new features
%are introduced one at a time during section \ref{sec:implementation}.

\subsection{Inferencing, Question Answering, and Dialogue Systems Today}\label{sec:comparison}

Generally, software that concerns itself with answering questions formulated in
ordinary human language falls in two main categories: \term{information
retrieval systems}, also known as \abbr{QA} (question answering) systems, and
\term{dialogue systems}.

The purpose of the former is in most cases twofold, although present \abbr{QA}
systems vary in the functionality they offer and results they actually deliver.
One goal is to enable their users to query for more complex information without
the need for a special query language or at least using a query language that
would be easy and natural to use even for the uninitiated user. The second
feature provides \abbr{QA} systems with another important distinction, viz. the
ability to utilize ontologies in order to relate query terms akin, but not equal
to words in target documents and thus deliver satisfactory results even for
fuzzy queries. This way the user can pose direct questions to a system instead
of having to search for a keyword to occur in the plain text of a document, like
they would have to do with traditional information retrieval systems. 

These systems typically do not try to \emph{reason} about
their input, but gain `knowledge' from applying statistical methods on large
input corpora. The methods used and results achieved vary throughout the
literature. One technique employed is feeding a database with large sets of
question/answer pairs, as described in \cite{rama} and then running basic
natural language processing tasks like part-of-speech tagging,
named-entity-recognition, syntactic and semantic parsing on the input.

Many systems are only able to deliver small bits of information, also known as
\term{factoids}, or just display fragments of their input text as it occurred in
their corpus. Their knowledge base is typically not interactively modifiable,
and acquiring knowledge and answering questions are two entirely separated
tasks.

\cite{webber}, argue that \abbr{QA} systems would undoubtedly benefit from
introducing inferencing to their capability set. They analyse some of the
typical \abbr{QA} tasks and give examples to show that \emph{reasoning},
i.e. logical inferencing on the unstructured text that underlies modern
\abbr{QA} systems would enhance their functionality.

Interestingly, early attempts at providing a software capable of answering
questions in natural language were mostly coupled with the usage of semantic
analysis rather than statistical methods.  \cite{chat} are among  the first to
attempt creating a question answering system with reasonable coverage. Their
effort has been focused on the geographical domain, and their implementation,
dubbed \pn{CHAT-80}, is written entirely in \prol. The lack of hardware
resources of that time forced the system to be written with efficiency in mind.
Moreover, it had to maintain a statical database of facts which was
hand-crafted. As such, it falls more into the category of question answering
systems than dialogue systems. Within its domain, \pn{CHAT-80} can deliver quite
impressive results for different types of questions, including questions about
the nature of relations between objects in its database.  This way, the system
can answer a user's question as \example{Which country is borders at least two
oceans?} without having a list of countries that border two oceans at its
disposal.

\cite{prontoqa} introduce the \pn{Pronto QA System}, which couples traditional
corpus-based \abbr{QA} techniques with deep semantic parsing, utilizing
\term{Discourse Representation Theory} as a data structure. Unfortunately, there
seems to be no working demo to date. The homepage of the ongoing efforts of
Johan Bos and his team can be found at the following address:
\href{http://svn.ask.it.usyd.edu.au/trac/candc/}{\url{http://svn.ask.it.usyd.edu.au/trac/candc/}}.

Discourse Representation Theory is also used in \cite{bosgabdsil}, who present
an approach to question answering to a certain extend similar to what is pursued
in this paper. They choose to pursue the theoretical background compiled by
\cite{gs:sqpa} and several later publications, including \cite{gs:q}. Moreover,
they also use external first-order inferencing tools to perform computations on
the consistency and informativity of the system's DRS models, which for this
purpose are first translated to first order logic.  Their results will be
discussed later on, and explicitly compared to the present results.

\cite{tenCate} present a tableaux method for resolving questions via inferenceand also
propose an algorithm in \prol{} as an implementation. Although their methods
seems promising, it was chosen to
the present approach will avoid the implementation of a specialized inferencing
system and rely on external inferencing tools.

Today it is clear that question answering systems make inroads to the search
market, and not many doubt that they are bound to eventually replace current
document/term-based search engines. Dialogue systems, are still not capable of
convincing a larger user base. Among other things, their representation of
knowledge and capability to interact with stored information is still considered
to be lacking.  Even though \abbr{QA} systems also operate on the domain of
answering questions, their approach to resolving them differs from that of
dialogue systems. The latter have a much higher obligation to correctness, since
their results are typically not subject to peer review and serve an entirely
different purpose.

\pn{Godot}, a very interesting project, is introduced in \cite{godot}.
\pn{Godot} is a so-called \term{embodied agent}, a robot with a sophisticated
software driving interaction with the outer world. As such its conception of
discourse not only relies on spoken dialogue but also also on sensory input, rather
than on an input corpus. Its knowledge is also \emph{dynamic}, i.e. it can be
extended through the aforementioned means, making it
dialogue system.
\pn{Godot} is focused on executing orders -- after evaluating their feasibility
in its particular discourse model -- and does so far not seem to support the answering
of questions about its state.

\curt{} is a prototype teaching system, designed to showcase the possibilities
of a full-fledged dialogue system. It differs from \pn{Godot} in some major
points, the most distinctive of which is that \curt{} only exists inside a
regular computer and has no access to sensory input. Also, \curt's
representation of knowledge differs from that of \pn{Godot} or the \pn{Pronto QA
System} in that it does not use discourse representation theory (yet). \acurt{}
addresses some of its shortcomings and lays ground to the implementation of a
broader feature set for it.

\subsection{Extending the Curt System}\label{sec:extension}

At the end of \cite{blackburnbos:cl1}, \curt{} is able to cope with a small
grammar and provides some features for interaction in natural
language. It does only understand basic extensional transitive and intransitive
verbs and its support for question answering is only rudimentary.
Anaphora resolution still remains an open issue.

In the course of this thesis, a new version of \curt{} is introduced, called
\acurt. This version is not only aware of its own state of knowledge, as was the
original \curt{} implementation, but can also gather information about the
knowledge and beliefs of individuals it knows about. Furthermore, \acurt{} is
able to account for several possible readings of a certain sentence at once and
in parallel, which is a side-effect of the introduction of the notion of
\term{possible worlds} into the system.

\acurt{} is able to parse and, according to the theoretical background of section
\ref{sec:theory}, correctly interpret sentences as the ones in
\ref{ex:whemba}-\ref{ex:whembc}.

\ex.
\a.  Mia knows (whether) Vincent snorts.\label{ex:whemba}
\b.  Mia knows who snorts.
\b.  Vincent believes that Mia likes Jody.\label{ex:whembc}

For this to work, the internal storage of \curt{} had to be restructured
significantly in order to deal with more than one possible world or situation.
Its grammar had to be enhanced so that it can handle \wh-embedding verbs
like \example{know} and \example{believe} and its inferencing functionality had
to be ported to the new system.

The next two sections explain, in order, the theoretical background and
practical operation of \acurt{}.

\chapter{Theory}\label{sec:theory}

The semantics of questions have traditionally been a very sophisticated field of
linguistics,  philosophy and logics, with many different approaches arguing for
and against each other. This section will not try to give an exhausting overview
of the field of question semantics since this would be a task of considerable
size and by far beyond the scope of this thesis\footnote{For a good summary of
some of the literature, particularly the issue of whether questions are a
concern of pragmatics or semantics, consider \cite[section 2 and 3]{gs:q}.}.
Instead, it was chosen to roughly follow the work by Groenendijk and Stokhof,
who present one of the best-established and recognized theories in the field.
In parts, the framework of the next sections is lending from other places, where
deemed appropriate or practical. As such, the theoretical framework of \acurt{}
is not going to be an exact realization of any particular approach to question
semantics, but rather an accumulation of interesting, and, most of all,
practicable concepts from different sources. The main sources, however, are
\cite{gs:q} and \cite{gs:sqpa}.

Following an overview of the terminology that will be used for the discussion of
the formal treatment of questions, an  elaboration on the differences between
different logical formalisms employed in the treatment of questions in natural
language sets the stage for the discussion of some theoretical frameworks and
issues. Then, the target domain of natural language is introduced by examples
alongside the mechanisms employed to interpret them. 

Note, however, that all choices presented in this section have been made
with the implementation of section \ref{sec:implementation} in mind.
Sometimes, theoretical soundness might suffer from these choices as the main
goal of the present work is to provide a proof-of-concept of a software that is
able to reason about possible worlds, employing first-order logic inferencing
tools in order to realize an approach to the treatment of embedded questions and
questions in general.

\subsection{Terminology}

Following \cite{gs:q}, the English word ``question'' is not used in its
traditional sense, but a distinction is made between \term{questions},
\term{interrogative sentences} and \term{interrogative acts} in the following
way:

\begin{termin}
  An \term{interrogative sentence}, or simply \term{interrogative}, is a
  type of sentence in natural language
  distinguished by certain characteristics
  \footnote{To quote \cite{gs:q}: \quotesf{[An
  interrogative sentence is] characterized by word order, intonation, question
  mark, the occurrence of interrogative pronouns}.},
  whereas an \term{interrogative act} refers to the utterance of an
  \term{interrogative}. Finally, a
  \term{question} will denote the semantic entity uttered by the
  \term{interrogative act}, and is introduced in section
  \ref{sec:protoq}
\end{termin}

Furthermore, interrogatives are split into several major categories.

\ex. \a. \label{ex:yesno} Q: Does Mia smoke? A: \emph{Yes.} or \emph{No.}
\b. \label{ex:const} Q: Who smokes? A: \emph{Mia smokes.} or \emph{Vincent
somkes.} or \emph{Nobody smokes.}

Example \ref{ex:yesno} is considered an \term{alternative question} or
\term{yes/no-interrogative} since only \emph{yes} and \emph{no} are possible
answers to this interrogative. \ref{ex:const}, on the other hand, is considered a
\term{\wh-question} or \term{constituent interrogative}, since one of the
sentences' nominal constituents is replaced by a \wh-pronoun, prompting an
enumeration of possible fillers for the resulting gap as an answer.

The following abbreviations are used: 
\abbr{Ty2}, for two-sorted type logic\footnote{See \cite{gallin:ty2} for a
discussion and \cite{gs:sqpa} for its usage in describing the semantics of
questions}, \abbr{IL}, for intensional logic\footnote{As used by
\cite{ptq}. See also \cite{gamut:2} for an excellent introduction to \abbr{IL}.},
\abbr{FOL}  is used as an abbreviation for first order logic. As an alternative
term for \term{possible world}, sometimes the expression \term{index} is used.

\section{Formalism: Models}\label{sec:formal}

Most of the traditional literature in natural language semantics follows
\cite{ptq} and thus also makes use of its underlying formalism,
\abbr{IL}. \cite{gs:sqpa}, however, use \abbr{Ty2} as it was
proposed in \cite{gallin:ty2}. \cite{z:ilty2} discusses the differences and
similarities in expressive power between \abbr{Ty2} and \abbr{IL} and comes to
the conclusion that the set of expressions contained in the former but not in
the latter might not be relevant for the treatment of natural language
semantics. Although he proposes a translation scheme between both languages,
translation from \abbr{Ty2} to \abbr{IL} seems to sometimes produce very complex
\abbr{IL} terms.

As was already mentioned in \ref{sec:curt}, \curt{} is based on the use of
external inferencing tools. These tools are only capable of treating first order
logic – an extensional framework. \prol{} on the other hand, is a language that
can very well make use of higher order formalisms for calculation\footnote{The
two most popular predicates in \prol{} used for implementing higher-order logic
are \code{call/n} and \code{apply/3}. This paper is based on \pn{SWI-Prolog}
which provides an implementation of the former, but not of the latter. In
addition \pn{SWI-Prolog} also defines several higher order aggregate functions,
including \code{maplist/4}, which will be used later on. See
\cite{naish:prolhio} for a discussion of higher order functional programming in
\prol.}, but as was already mentioned, the implementation of separate
inferencing mechanisms for \acurt{} will be avoided. While there exist
inferencing tools for modal logic\footnote{ The interested reader is referred to
\pn{Molle} (\url{http://molle.sf.net}), a modal theorem prover that comes with a
graphical user interface which allows interactive investigation of the
constructed indices.}, a restructuring of the \curt{} system to take advantage
of them would be beyond the scope of this thesis.  It was therefore decided to
avoid modal logic as much as possible and implement a pragmatic approach to
representing indices in order to increase the expressive power of the system.

Most research on the logic of questions (sometimes also referred to as eroteric
logic) suggests that the treatment of interrogatives demands its own formalism.
\acurt{} employs a formalism that departs from both \abbr{IL} and \abbr{Ty2} in
some ways. Most notably, the notion of a \term{model} is changed as follows.

\begin{definition}\label{Models}
  \stress{Models\footnote{See also the Appendix, section \ref{sec:examplemodel}
  for an example of such a model.}}
  \begin{enumerate}
    \item A model $M$ is a set of pairs of \term{indices} $i$ and \term{possible
    worlds} $w$: $\pair{i,w} \in M$. 
    \item A world $w$ at index $i$, $\pair{i,w} \in M$, is a triple of $D$, the set
    of individuals, $F$, the set of functions, and $\varphi$ a first-order
    logical formula that holds in this particular model: $w = \pair{D,F,\varphi}$
    \item A \term{function} $f \in F$ is one of the following: 
    \begin{itemize}
      \item $f(a,\sigma,L)$, where $\sigma$ is the function symbol, $a$ denotes
      the function symbol's arity and $L$, the set of arguments, for which the
      following holds: $|L| = a$,
      $L = \set{\pair{x_1,\ldots{},x_a}|x_1,\ldots{},x_a \in D}$
      \item The special case $f(0,\sigma,c)$, where $c \in D$, which denotes the
      assignment of a proper name $\sigma$ to $c$.
    \end{itemize}
    \item $\pair{i,w}, \pair{j,w'} \in M, w = w'$ iff $w = \pair{D,F,\varphi}, w' =
    \pair{D',F',\varphi'}$ and $\varphi = \varphi'$.
    \item If $W \in M$, then we say $\forall W$ quantifies over all members of $M$.
    \item If $W \in M$, then we say $\exists W$ quantifies over the existence of
    one member of $M$, so $|M| \geq 1$ and $W \in M$.
  \end{enumerate}
\end{definition}

One of the more important differences to the `traditional' \abbr{Ty2} notion
of a model, is that it was decided to assign every world a unique domain of
discourse referents or \term{individuals}, instead of having one global set of
individuals. This follows the definition of
\cite{gs:q}, and is generally referred to as \term{partitioning of the logical
space}. \cite{tenCate} propose a
quantification over the assignment function $g$ in order to generalize a
partitioned logical space into a more standard \term{Ty2} model, however, this
would also demand the introduction of an set of an interpretation functions $I$
to assign elements of $W$ an extension, which is not necessary here because
every possible world is, in fact, an extensional model.

Note that there is no requirement $\pair{i,\pair{D,F}}, \pair{j,\pair{D',F'}}
\in M \wedge D\neq D'$, and most of the time, a model's possible worlds will
indeed share many of their individuals or have entirely equal domains, but
diversion in the world's domains will sometimes be useful and unavoidable.
%fixme: really?

Moreover, every possible world carries a formula $\varphi$, to which $\pair{D,F}$
is the minimal extensional model. This justifies the choice to define equality
of possible worlds in terms of equality between the logic expressions $\varphi$
they carry. This is also necessitated by the fact that \acurt{}  only reasons,
i.e. runs its reasoning tools on these formulae and can therefore not make a
difference between possible worlds in any other terms. $\pair{D,F}$ will serve
only as a ground for answering questions. This entails, that the traditional
difference between \term{models} and \term{readings} present in \curt, is not
carried over, but instead the readings are now constituents of a model.

Defining quantification over world-index pairs enables \acurt{} to make a
distinction between \term{exhaustive} and \term{partial answers}. See section
\ref{sec:answerhood}.

Finally, logic expressions in \acurt{} do not typically carry any kind of type,
since \prol{} does not employ a type system. The grammar rules of \curt{} (which
\emph{do} have \stress{syntactic} types) ensure a correct application of
unsaturated formulae. These formulae are for the most part expressions of \pn{FOL}
with two enhancements, introduced in section \ref{sec:protoq} and \ref{sec:epistemic}

\section{On the Proper Treatment of Questions}

The analysis of the meaning of questions poses an interesting task to modern
research on natural language semantics. It has so far attracted many different
theories on how to represent and compute with them.
But as interesting topics go, the semantics of questions
has attracted many different views on this issue.  It is important to first
settle on what constitutes and characterizes a \emph{direct} question before one
can turn to the treatment of \emph{embedded} questions and propositions. Most
literature seems to agree that the latter is but a special case of the former,
even though sometimes the notion of a direct question is abstracted to a
\term{proto-question}.


\subsection{The Question of Domain}

Often question semantics has been reduced to a problem of pragmatics, treating
direct interrogatives as nothing more than a paraphrase of embedded
interrogatives.\cite{tichy} even goes as far as claiming equality in semantic
(truth) value between indicatives and (embedded) interrogatives. His analysis
permits omitting a new the introduction of a new phrasal category into the
logical formalism that would otherwise be needed to account for questions.
Interrogative sentences then merely infer a different `attitude' in the
speaker's utterance. Putting this attitude into context and making it yield the
correct results is then left to pragmatics, which treats them as interrogative
acts in the setting of a dialogue.

The issue is ultimately whether the answering of questions belongs in the realm
of pragmatics or semantics. Tichy's approach, along with other, similar ones,
has been challenged by \cite{gs:q}, who argue for a treatment of interrogatives
on a semantic level, conceding a \emph{meaning} to interrogatives.  They also
propose certain adequacy criteria for theories of the semantics of questions in
order to be able to compare different approaches made in the literature and also
to defend their own solution. Their analysis is based on the introduction of
another type of logical expression, a \term{question}, which embeds a
proposition (with possibly one or more free variables) under a question operator
\term{?}.  In contrast to indicatives, questions do not yield a truth value, but
are defined by means of their \term{possible answers}.  They are formally
introduced in section \ref{sec:protoq}.

%\cite{karttunen:1977} attempts a treatment of
%questions in a similar manner, but also \emph{denies} a difference in logical
%type for questions as such. According to his hypothesis, direct questions are a
%syntactic variant of indirect (embedded) questions and yes/no interrogatives are
%a ``syntactically degenerate variant'' of constituent interrogatives.

%\begin{quotes}A direct question can be treated as semantically equivalent to a
%certain kind of declarative sentence containing the corresponding indirect
%question embedded under  a suitable `performative' verb.\end{quotes}

%Using this strategy, Karttunen escapes the treatment of direct  questions and
%thus gives an approach for indirect (i.e. \emph{embedded}) questions  only.
%In this respect, his analysis is somewhat similar to the one presented in
%\cite{gs:sawhq}, but the latter goes on to challenge certain aspects of his
%hypothesis.

\cite{blackburnbos:cl1} treat interrogatives in a similar manner as \cite{gs:q},
using a special operator. From chapter six of \cite{blackburnbos:cl1}:

\begin{quotes}
  Questions, unlike assertions, don't have truth-values, so it would be
  misleading to represent them as ordinary formulas.
\end{quotes}

Their notion of questions was modified for \acurt{} in order to account for
yes/no interrogatives and larger sets of answers made possible by using a model
representation which allows for possible worlds.

\subsubsection{The Grammatical Domain of \acurt}

Constituent interrogatives further differ in \emph{what} they ask for. One can
assume the existence of at least three different types:

\begin{enumerate}
  \item NP-questions ask for a noun phrase, e.g. \example{Who shot Vincent?
  Butch (shot Vincent.)} or even more then one noun phrase at the same time, \example{Who had
  what in the hash bar?}\footnote{Multiple constituent interrogatives pose a
  somewhat greater challenge, since they come with an inherent ambiguity, the
  so-called \term{Baker Ambiguity}, as presented in \cite{baker:1970}.
  Such ambiguities will not be concerned here.}
  \item PP-questions ask for prepositional constituent, e.g. \example{Where did
  Jules stay? In Amsterdam.}
  \item VP-questions ask for an action performed by an individual, e.g.
  \example{What did Butch do? He shot Vincent.}
  \item Temporal questions ask for the date of a certain event, e.g.
  \example{When did Butch shoot Vincent? When he entered the toilet.}
  \item etc\ldots
\end{enumerate}

This classification of interrogatives can go on for quite some more entries,
including local, modal, causal and many sorts of other questions. Out of all
those categories, only NP-questions are be supported by \acurt{}, with an option to
integrate other types of interrogatives at a later stage.

\subsubsection{Pragmatic Concerns}

Another important distinction in the definition of questions, is the definition
of \term{answerhood}, i.e. what constitutes a valid answer to a given
interrogative. The different positions on this topic are elaborated on in
section \ref{sec:answerhood}. Regardless of the concept of an optimal answer, it
is clear that upon finding an answer, said answer has to be \emph{realized} in
natural language using text or speech synthesis. When formulating an answer,
many pragmatic considerations have to be taken into account.  However, as the
uttering of an embedded question does not prompt an answer\footnote{Unless, of
course, the question is uttered as an \term{imperative} -- but this is an
entirely different class of problems that is not discussed here.}, the present
work will not directly concern itself with a solution to this problem.

The remainder of this section will mostly compare the first (purely extensional)
approaches by \cite{blackburnbos:cl1} with the more general framework of
\cite{gs:q} and come up with a representation of the latter within the lines of
the implementation of the former. It will assume a modal logic as was discussed
in section \ref{sec:formal}, but an effort is made to isolate the reasoning on
possible worlds in just a few rules, so the rest of the framework can stay
extensional and thus benefit from \curt's external reasoning tools and present
internal mechanisms.

\subsection{The Semantics of Interrogatives: The ? Operator}\label{sec:protoq}

\pn{Helpful Curt}, the most advanced version of \curt{} \cite{blackburnbos:cl1}
present, only addresses the problem of direct constituent interrogatives and
does not treat yes/no interrogatives or embedded questions. The interrogative
operator, as it will be introduced shortly, is designed to cover the semantics
of both types of questions simultaneously. It takes a proposition (a formula
yielding a truth value as a result) and yields a set of propositions. \acurt{}
represents embedded questions by embedding a special construct, a
\term{proto-question}, under verbs of certain types (discussed in section
\ref{sec:ontology}). Since the yes/no interrogative is considered a special case of the
constituent interrogative the semantic formulae produced by the grammatical back
end of \acurt{} show no difference in structure for the two cases.

\begin{definition}\label{def:qop}
  \stress{The Question Operator}, following \cite{gs:q}
  \begin{enumerate}
    \item If $\varphi$ is a proposition in which the free variables
    $x_1,\ldots,x_n$ for arbitrary values of $n \geq 1$ occur,
    $?\pair{x_1,\ldots,x_n}.\varphi$ denotes a \term{question}, associated with
    a constituent interrogative.
    \item If $\varphi$ does not contain any free variables, $?.\varphi$ is
    considered a question associated wit a yes/no interrogative.
    \item * \label{def:outermost}The question operator does \emph{not combine} with any other operator.
  \end{enumerate}
\end{definition}

To give two examples of the use of the question operator:

\ex. \a. Who shot Vincent?
\b. $?x.shot(x,vincent)$

\ex. \a. (Does) Vincent snort(s) or smoke(s)?
\b. $?.snort(vincent) \vee smoke(vincent)$

In the latter example, English grammar in written form will demand the phrase to
contain the auxiliary verb \emph{does}, although spoken form would permit
\example{Vincent \stress{snorts}?} as a question with a certain intonation
pattern on the verb. This is not of relevance to the treatment of embedded
questions, where the auxiliary is not needed.

Note that the last clause of definition \ref{def:qop} enforces the question
operator to always be the last operator of a formula. This constraint will be
overridden during parsing of questions to produce forms that do have nested
questions.  When calculating the models and readings of a formula, however,
these operators will be resolved first. This is done in order to increase the
efficiency during parsing, since the syntactic parser will sometimes have to
backtrack. In fact, the generation of \term{logical forms} and their
interpretation are always two distinct steps in \acurt.

As was already mentioned, \pn{Helpful Curt} has a certain, albeit limited
support for evaluating NP constituent interrogatives. Blackburn and Bos adopt a
sort of pseudo-notation of the question operator in what they call an
extra-logical form. It does not yield a truth value, but is internally evaluated
in a special case.

\curt{} sees a question as a \emph{quantification over a certain individual
$X$}, for which two conditions, $R$ and $S$, must hold. \cite{bosgabdsil} call
$R$ and $S$ the \term{domain} and \term{body} of the question. The domain of a
question is usually ontological information, posing a restriction on the
quantified variable. It is selected according to the \wh-pronoun that triggered
it. So \example{who} prompts the domain to contain $person(x)$, whereas
\example{which}, used as a quantifier in a phrase as \example{which boxer}, would
impose the restriction $boxer(x)$. \ref{ex:bbqueexa} and \ref{ex:bbqueexb} give
an example of how \curt{} parses constituent interrogatives.

\ex.\label{ex:bbqueexa}
\a. Which boxer likes Jody?
\b. $?x.boxer(x) \wedge like(x,jody)$
\b. \label{ex:bbqueparsea} \code{que(X, boxer(X), like(X, jody))}

\ex.\label{ex:bbqueexb}
\a. Who shot Vincent?
\b. $?x.person(x) \wedge shoot(x, vincent)$
\b. \label{ex:bbqueparseb} \code{que(X, person(X), shoot(X, vincent))}

Similarly, yes/no interrogatives, which are new to the \curt{} system, are
parsed using the same sort of operator:\footnote{The actual representation in
the code will differ a little in order to work around a bug in the system.}

\ex.
\a. (whether) Vincent snorts.
\b. $?.snort(Vincent)$
\b. \lstinline!que([], [], snort(vincent))!

\ex. \a. (whether) Mia or Jody dances.
\b. $?.dance(mia) \vee dance(jody)$
\b. \lstinline!que([], [], or(dance(mia),dance(jody)))!

The aforementioned special case in \pn{Helpful Curt} only evaluates a question
against the extensional model which is used as a representation of \curt's state
and knowledge. In this it differs substantially from the approaches made by
Groenendijk and Stokhof and also later approaches by \cite{g:is}. \acurt{}
aims to bridge the gap between both worlds.

\subsection{What Answers the Question?}\label{sec:answerhood}

Knowing the structure of an interrogative, the next step is to evaluate the
question's \stress{value}, namely the propositions it generates. This will
invariably bring up the question of the definition of \term{answerhood}, viz.
the relation of the question to its answer or answers with respect to the
current discourse. Note that an answer other than `no' or
`nobody' cannot occur without a certain discourse or background knowledge already being present
for the answer to be evaluated.

\cite{hamblin:q} defines a basic notion of answerhood, relating the value of
a question with the set of propositions that might serve as partial answers to
it. \cite{gs:q} phrase Hamblin’s postulates as follows:

\begin{quotes}
  \begin{enumerate}
    \item An answer to a question is a sentence, or statement.
    \item The possible answers to a question form an exhaustive set of mutually
    exclusive possibilities.
    \item To know the meaning of a question is to know what counts as an answer to
    that question.
  \end{enumerate}
\end{quotes}


For yes/no interrogatives this means, that if the question is $?.\varphi$, the
answer to it is either $\varphi$ or $\neg\varphi$. For constituent
interrogatives, this question is not so easily answered.  There is one central
issue to be addressed first: should the set of answers be an \emph{exhaustive}
set of answers, including the negative ones, or just the set of possible
\emph{true} answers?

The following situation might serve as an example: Suppose Mia and Vincent both
play air guitar. Butch and Marsellus do not play air guitar. When answering the
question \example{Who plays air guitar?}, one can observe three degrees of
exhaustiveness, according to \cite[section 1.5]{gs:sawhq}. Clearly, it does not
suffice to just say \example{Vincent.} or \example{Mia.}, since that would still
not account for all possible answers to the question, and that both
\example{Butch.} and \example{Marcellus.} would be outright wrong. In the
analysis of \cite{karttunen:1977}, every set of propositions that is a superset
of the set of propositions containing \example{Mia plays air guitar.} and
\example{Vincent plays air guitar.} would be a valid answer. This is the lowest
degree of exhaustiveness, and accounts for all \emph{true} answers. So, if
Yolanda believed to not only know of Vincent and Mia that they played air
guitar, but also of Butch, she would still know all possible true answers, and,
according to this analysis, have a correct answer.

\cite{gs:sqpa} argue for two more degrees of exhaustiveness. The first
one requires the answer not only to contain all possible \emph{true}
propositions, but also to \emph{not contain any \stress{false} propositions}. So
in this case, if Yolanda believed Butch played air guitar, she would not know
the true answer.

\footnotetext{
There even more ways of categorizing the number of answers to a question. So far
the so called \term{mention-all} case has been discussed, but when considering
a sentence like \example{John knows where to buy a Chinese newspaper.} (example
due to \cite{rooy:qar}), it is clear that only one or two out of all possible
propositions for \example{Where to buy a Chinese newspaper?} will be sufficient
for John to know in order for him to know where to actually buy one. This has
been regarded as the \term{mention-some} reading.}

The second and strongest degree of exhaustiveness 
requires any set of answers not only to contain the set of propositions that
hold, but also the set of propositions that \emph{do not hold}. In this case, for
the sentence \example{Yolanda knows who plays air guitar.} to be true, the
sentence \example{Yolanda knows who does not play air guitar.} would
automatically follow.


\subsubsection{Combinatorial Problems}

Thus the maximum degree of exhaustiveness bears several problems. The most
obvious one is that of a considerable \emph{combinatoric explosion}\footnotemark. In the
previous example, containing only four individuals, the question
$?x.play\_air\_guitar(x)$ evaluates to a total set of $2^4$ propositions:
\example{Nobody plays air guitar.}, \example{Only Butch plays air guitar},
\example{Only Mia plays air guitar},\ldots{}, \example{Only Butch and Mia play
air guitar},\ldots{}, \example{Everybody plays air guitar}. Generally, every
question will evaluate to $2^{|D|}$ propositions. As \cite{bosgabdsil} show, to
find a suitable answer to a given partition of logical space (i.e. a
\emph{question}), the logical inference tools would have to perform consistency
checks on the union \emph{every single} possible answer with the discourse.

\footnotetext{
There is at least one more problem with this analysis although \cite{gs:q} hint
at an easy solution for this. Suppose Yolanda does not
know Butch, but she knows Mia and Vincent, and that they play air guitar. But since
she does not know about Butch, she has no chance of knowing a \emph{most
specific} answer. Groenendijk and Stokhof's analysis thus requires every
individual to have exhaustive knowledge about the discourse domain.}

\cite{bosgabdsil} go on to propose a method for taming the computational
complexity of Groenendijk and Stokhof's analysis, but they do so at the cost of
a strong account for exhaustivity. Their system reduces the size of the
partition to essentially four propositions: \example{Everybody plays air
guitar.}, \example{Somebody plays air guitar.}, \example{Somebody does not play
air guitar.}, \example{Nobody plays air guitar.} Now the inferencing task has
shrunk down considerably, but it does not even require \emph{every true answer}
to be known in the case that the body of the question does not hold for at least
one individual, which will, for non-trivial domains, almost always be the case.

As \acurt{} will only have to deal with answers to questions
internally, it will also only regard maximally exhaustive interpretations of
answers.

Furthermore, correctly answering a direct question with respect to relevance and other
pragmatic factors is a task left as a task for turning \curt{} into a complete
dialogue system more potent of dealing with queries to its knowledge.

\section{Embedded Questions}\label{sec:embed}

After having agreed on a treatment of interrogatives, the next step is to
construct a framework for embedding the result under certain verbs.  This
section provides a classification of embedding verbs and elaborates on the
details of the relationship between embedded and embedding constituents of an
utterance.

\subsection{Syntax}
As was already mentioned, clause \ref{def:outermost} of definition \ref{def:qop}
will be overridden when treating embedded interrogatives. The following example
illustrates a syntactic parse of an embedded interrogative:

\ex. \label{ex:whembed}
\a. Mia knows who snorts.\footnotesize
\b.  \Tree
[.$t$\\$know(?X_1.snort(X_1))(Mia)$ [.$T$ { $\lambda X.X(Mia)$\\Mia } ]
[.$IV$\\$\lambda X.know(?X_1.snort(X_1))(X)$ [.{$IV/\bar{t}$} 
{ $\lambda S.\lambda X.[  \lambda Y.[ know(Y)(X)](S) ]$\\knows } ]
[.{$\bar{t}$\\$?X_1.snort(X_1)$ } [.$\bar{t}/t$ ? ] [.{ $t$\\$snort(X_1)$ }
[.$T$ { $X_1$\\who } ]
[.$IV$ { $\lambda X.snort(X)$\\snorts } ] ] ] ]!\qsetw{5cm} ]\normalsize
\b. $know(?X_1.snort(X_1))(Mia)$\label{ex:formula}

Note that this example is using  the (syntactic) type system of \cite{gs:sawhq}
but different meaning postulates. Grammar rules, later explained in
\ref{sec:curtgrammar}, are
responsible for determining the correct order of functional application in the
untyped logical system \acurt{} is using.

One can see \ref{ex:formula} as being an \emph{intensional} denotation of the
meaning of \ref{ex:whembed} rather than an extensional one. In fact, this
sentence will evaluate to more than one proposition in case there is more than one
person snorting.

The reason for $?$ to be embeddable within more complex formulae is not merely one
of convenience, since it cannot be evaluated without access to a particular
world's set of individuals $D$. This would mix semantic interpretation with
the parsing process and such side-effects were not deemed desirable.
Instead, before calling its inferencing tools \acurt{} will look out only for
the question operator itself and take appropriate measures.

Another advantage of this approach is that the treatment for yes/no
interrogatives and constituent interrogatives will not require special
intervention at the level of grammar.

\ex. \label{ex:simpleembed}
\a. Mia knows whether Vincent snorts.  \footnotesize
\b. \Tree
[.$t$\\$know(?.snort(Vincent))(Mia)\wedge ?.snort(Vincent)$ [.$T$ { $\lambda X.X(Mia)$\\Mia } ]
[.$IV$\\$\lambda X.know(?.snort(Vincent))(X)$ [.{$IV/\bar{t}$} 
{ $\lambda S.\lambda X.[  \lambda Y.[ know(Y)(X) ](S) ]$\\knows } ]
[.{$\bar{t}$\\$?.snort(Vincent)$} [.$\bar{t}/t$ whether ] [.{ $t$\\$snort(Vincent)$ } 
[.$T$ { $\lambda X.X(Vincent)$\\Vincent } ]
[.$IV$ { $\lambda X.snort(X)$\\snorts } ] ] ] ]!\qsetw{5cm} ]
\normalsize
\b. $know(?.snort(Vincent))(Mia)$

The question operator in front of saturated formulae ($?.\varphi$) could also be
rephrased as

\[ \forall w\in W_M [ \varphi(w) \wedge \neg\varphi(w) \]

Obviously, the partitions for yes/no interrogatives is much smaller than it is
for constituent interrogatives and it does not need access to a particular model
in order to form its partition. Yet this is not reason enough to give up the
more general approach already described.

\subsection{Types of Facts, Assertions and Interrogatives, Towards an
Ontology of Embedding}\label{sec:ontology}

Embedded clauses are always embedded under a certain verb, which usually
takes a certain type of phrase and returns an intransitive verb. Examples of
this were given in the previous section in \ref{ex:simpleembed} and
\ref{ex:whembed}. The verb used in both examples was \example{know}.

Examining some examples of other verbs, one will quickly establish that not
every verb can embed any kind of sentence, but also that there seems to be  a
pattern one might be able to exploit. Leaving the question operator in place
again helps distinguishing between the different types of embedding in such
cases, since it is of major importance to distinguish between indicatives and
interrogatives, as we can see from the next example.

\begin{itemize}
  \item Embedded yes/no interrogative
  \ex. \a. Mia \emph{knows} whether Vincent snorts.
  \b. * Mia \emph{believes} whether Vincent snorts.
  \b. Mia \emph{asks} whether Vincent snorts.

  \item Embedded constituent interrogative
  \ex. \a.  Mia \emph{knows} who snorts.
  \b. * Mia \emph{believes} who snorts.
  \b. Mia \emph{asks} who snorts.

  \item Embedded proposition
  \ex. \a. Mia \emph{knows} Vincent snorts.
  \b. Mia \emph{believes} Vincent snorts.
  \b. * Mia \emph{asks} Vincent snorts.

\end{itemize}

These examples show that there is no difference between
constituent interrogatives and yes/no interrogatives concerning the syntactic
position they can occupy. Furthermore, it seems that the three different verbs
behave differently: `know' can take both, questions and propositions, `ask' can
only take questions, and `believe' can only take propositions.

\cite{karttunen:1977} introduces means of classification for \wh-embedding
verbs, the nature of which \cite{lahiri:diss} covers in a much broader scope.
Both come to the conclusion that different verb classes ask for different
strategies of treating the \wh-clauses they embed and also pose different
restrictions on the type of clause they expect. The present work will mostly
draw from the results of \cite{ginzburg}, who summarizes previous efforts,
in order to provide an account for an ontology of embedding verbs.

According to \cite{ginzburg} there are (at least) three different types of
embedding verbs\footnote{The nomenclature of the different types departs from
what Ginzburg presents. He calls Questioning predicates QI, propositional
predicates TF. Resolutive and factive predicates are the same.}:

\begin{definition}\label{def:embed1}
  Types of Embedding Predicates
\begin{itemize}
  \item \term{Questioning predicates}, take only questions as counterparts.
  Examples: \example{ask}, \example{wonder}, \example{investigate},
  \example{discuss}
  \item \term{Propositional predicates}, need propositions as counterparts.
  Examples: \example{claim}, \example{believe}, \example{think},
  \example{assume}, \example{deny}
  \item \term{Factive predicates} take both, questions and propositions.
  Examples:
  \example{know}, \example{forget}, \example{reveal}, \example{discover},
  \example{report}, \example{predict}
\end{itemize}
\end{definition}

Both, \term{questioning predicates} and \term{propositional predicates} do not
effect any change or demand any particular state in the current situation of the
world in order to be used. One can ask \example{Who dances?} without anybody
present being in the state of dancing and one can believe someone to be in the
state of dancing, without this actually being the case. They are clear in what
they take as a counterpart, too: one cannot \emph{claim who dances} and one
cannot \emph{ask every boxer dances}.

The matter is more complicated with \term{factive predicates}, where
one might wish to establish another distinction. Recall the example from section
\ref{sec:answerhood}. In order for Yolanda to \emph{know} who plays air guitar,
her assumed knowledge and the real world have to agree on this fact. If Yolanda
thought Butch played air guitar, but in reality he did not, clearly Yolanda
wouldn't know who plays air guitar. She would only \emph{believe to know}.
This also holds for
verbs like \example{discover}, \example{reveal} and \example{forget}. But does
it also hold for \example{report} and \example{predict}? Although both verbs are
in the same \emph{syntactic} category, since one can \example{predict/report who will
win/wins the elections} and also \example{predict/report that a man will
win/wins the elections}, they seem to demand different \emph{semantic}
treatment, since both, predictions on, and reports of the results of elections are
in no way guaranteed to be true.

So, another distinction has to be made. Restating the last clause of definition
\ref{def:embed1}:

\begin{definition}\label{def:embed2}
  Resolutive and factive predicates:
  \begin{itemize}
  \item \term{Factive predicates} take both, questions and propositions, and
  demand the extension of (the set of) propositions they embed to evaluate to true (exhaustively).
  Examples:
  \example{know}, \example{forget}, \example{reveal}, \example{discover},
  \item \term{Resolutive predicates} take both, questions and propositions, and
  do not demand the extension of (set of propositions) they embed to have any particular truth value.
  Examples:
  \example{report}, \example{predict}
  \end{itemize}
\end{definition}

Yet still, all is not solved. The following example, taken from \cite[example
(27)]{ginzburg} illustrates a flaw in the definition of factive predicates:

\ex. 
\a. Jill believed a certain hypothesis. Hence, Jill believed that hypothesis
to be true.
\b. \label{ex:ginzburg}Jill discovered a certain hypothesis. It \emph{does not follow} that Jill
discovered that hypothesis to be true. 

\cite{ginzburg} defends an analysis that
introduces another distinction in the nature of indicatives: the distinction
between \term{facts} and \term{propositions} -- where in \ref{ex:ginzburg} the
embedded clause is treated as a proposition and not as a fact.

However, since \acurt{} is only concerned with embedding \wh-sentences, and
because it would exceed the scope of this thesis to introduce another type of
indicative, only the four already explained types of embedding verbs are 
considered here and a modification of the notion of indicatives will be left for
further research.

\subsection{Modeling Knowledge}\label{sec:epistemic}

So far, it has been suggested that questions can be interpreted as partitions of
the logical space of the current discourse model. Furthermore, a means for
evaluating questions has been presented, again, with respect to the current
discourse model. Then these questions, can be embedded under certain verbs in
certain ways. But what is the assertion a sentence as \example{Mia knows who
snorts.} makes?

When treating issues like this, the link between philosophy and language becomes
most obvious. One can assume that the knowledge about the world will differ,
sometimes greatly, between different individuals. One will also have to assume
that every person has their own beliefs. In \emph{knowing} or \emph{believing},
or even \emph{imagining}, one can create an image of what it could possibly look
like.

This concept of \term{possible worlds} probably dates back to Leibniz' theory,
who assumed that we live in the best of such possibilities. \cite{lewis} gives
an elaborate description of what the construction of such worlds might look like
and what their nature is -- or might be. He states that every individual
holds two sets of \term{epistemic} and \term{doxastic} alternative worlds. The
former is what an individual thinks is true, the latter accounts for its
beliefs.

\wh-embedding verbs embed sets of propositions, and they establish a relation
between those propositions and their subject, one or more individuals. Depending
on the type of verb, it can be that this individual wonders about a certain
fact, reports it,  knows about it or believes it. In the latter two cases,
something more happens: the proposition is asserted to be a fact of the
individual's \emph{epistemic} or \emph{doxastic} worlds, which are worlds, just
like any other possible world.

Discussing whether knowledge has to be \emph{true}, for any particular
definition of truth, or whether beliefs have to be consistent, or even if an
individual has to know about itself in its own alternative worlds, is a moot
point from a computational perspective. Suffice to say, that \acurt{} treats its
own models as representing the truth. Furthermore, it introduces two new sets of
worlds, accounting for epistemic and doxastic possibilities, with respect to
worlds of its own model and individuals thereof. The sets $\mathcal{E}$ and
$\mathcal{D}$ are thus defined in the following way:

\begin{definition}
  Epistemic and Doxastic Models

  An epistemic or Doxastic model is a triplet \pair{d,i,w'}, where
  \begin{itemize}
    \item i is a reference to a possible world $w = \pair{D,F,R}$ of the main
    model of the system, $\mathcal{M}$
    \item $d \in D$
    \item $w'$ is a possible world, defined the same way as before: $w' =
    \pair{D',F',R'}$
  \end{itemize}
\end{definition}

Whenever the relation of knowledge between an individual and a proposition is
encountered, the system records it within the epistemic worlds of the
individual it holds for, also noting the indexes of the possible worlds the
proposition holds in. Almost the same will be done for beliefs, which end up in
the doxastic worlds of an individual, but are not required to be true of a
possible world in \curt's knowledge.

\acurt{} is an experimental program. As such, it can serve as a ground for
experimentation to find out if this definition is better suited, or any other,
for representing the \example{true} state of affairs.

%\section{Treating Disjunction}

%Consider the following example dialogue with the \pn{Helpful Curt}
%system\footnote{In a real conversation with \curt, it will usually inform the
%user about the activity of external inferencing tools. This has been omitted and
%marked by \code{[…]}.}:

%\begin{verbatim}
  %> Mia or Jody dances.
  %[…]
  %Curt: OK.

  %> Who dances?
  %[…]
  %Curt: This question makes sense!

  %Curt: maybe mia
%\end{verbatim}

%Clearly, \curt{} should give a full answer in this situation, but instead it
%just guesses, and does not reply with something along the lines of \example{Either
%Mia or Jody}. Unfortunately, this result does not suffice for building a system
%that can properly treat embedded questions, since it does not reflect all
%available information about the discourse properly.

%A solution to this will again require the presence of the concept of
%\term{possible worlds}. When uttering the phrase \example{Mia or Jody dances.}
%then this can be understood as creating \emph{two possible scenarios}, one in
%which there is an individual named \example{Mia}, and this individual dances,
%and one in which there is an individual named \example{Jody} and the same holds
%for this individual. In effect, this can be translated to two possible worlds
%being present in the current discourse.

%\begin{definition}
  %Disjunction

  %For any formula $\rho$ holding a disjunction, this formula can be expressed as a set
  %of the constituents of the disjunction

  %\[ \rho = \varphi \vee \psi, \textrm{ then } \rho' = \set{\varphi, \psi}\]

  %%Every other operator or function can take a set of propositions of the type of
  %%$\rho'$ to produce a set of propositions containing \emph{all possible
  %%permutations} of $\rho'$. For every possible operator $\sem{$\cdot$}$ with arity
  %%$n$ and arguments $\sem{$\cdot$}(\set{\varphi,\psi})$

  %todo: more generic: just go from 1 to j, which is rho, then from j+1 to n

  %\begin{eqnarray*}
    %%\sem{$\cdot$}(\rho') \mapsto \set{\varphi,\psi}\\
    %%\sem{$\cdot$}(\sigma_1,\dots,\sigma_j,\rho )
    %\varphi
    %%\sigma \wedge \set{\varphi, \psi} \mapsto \set{\sigma\wedge\varphi,\sigma\wedge\psi}\\
    %%\sigma \to \set{\varphi, \psi} \mapsto \set{\sigma\to\varphi,\sigma\to\psi}\\
  %\end{eqnarray*}
%\end{definition}

\section{Inquisitive Semantics}\label{sec:altq}

The very latest research now focusses on providing an adequate logical semantics
for treating questions and even discourse within a dialogue. Groenendijk, in his
theory he calls ``Inquisitive Semantics'', argues for a completely new of of
treating dialogues (``Dialogue Management'') -- which naturally includes a
treatment of questions. His work on this topic has been prepared in \cite{g:is}
and is further developed in the notes to his course held at ESSLLI in Hamburg
(\cite{g:isdm}), just two weeks before work on this paper was to be finished.
Because of this time constraint, and because the latter paper is to date an
unfinished document, his theory could not be considered as a ground for the
implementation of \acurt. His findings, however, may very well serve as a ground
for further research on dialogue systems.

\subsubsection{The Rest}

\prol{} programs can be self-modifying, or dynamic. In this way the program is
able to change its own state and thus store information.

\subsubsection{Disjunction and Ambiguity}
Consider the following sentences:

\ex. \label{ex:disjunction} \a. Mia or Jody dances
\b. Either Butch snores or Vincent is using the chainsaw

\pn{Helpful Curt}, with its unpartitioned logical space and its lack of
the awareness of more than one (possible) world, would generate only one model
to represent the facts of \ref{ex:disjunction}. It is therefore only able to
generate \emph{one} possible answer to both questions, effectively accounting only
for one possible constituent of the disjunction, prefixing it with a `maybe' to
indicate its uncertainty.

Another example where \curt{} has to sacrifice exhaustiveness over its one-world
representation of the world is given in \ref{ex:ambig}

\ex.\label{ex:ambig} Every boxer likes a woman.

This will generally generate two readings, but only one model -- with only one
world. And further discourse will have to choose between one of the readings and
stick with the minimal model constructed by the model builder. This is not
really satisfactory and extending the dimensions of knowledge within \curt{} seems
necessary in order to treat such examples.

The previous work on \acurt{} enables it to generate a \term{choice point} -- and
moreover, not pursue only one path of it, but both at the same time, by
generating more than one world in its model.
Unfortunately, this will build up very large models over time and computations
with it might prove to be very complex and resource demanding. Adding such a
functionality to the system, however, is not a hard task and hopefully the
technique employed here can be further refined. %todo: maybe add a footnote
%about future prospects or so

We first derive know(john,all(X,imp(woman(X),love(X,john)))) from the syntax of
john knows whether every woman loves john, then we see if
all(X,imp(woman(X),love(X,john))) is valid (uninformative) and ridicule John, if
it's not valid and not consistent, assert his knowledge of the fact that it's
not true. If it's true, we assert he knows that.

For whether we can ridicule him if it's not true (inconsistent).
The MB/TP do a comparison \Disc to all(X,imp(woman(X),love(X,john)))

An important observation to be made here is that only \emph{assertive} verbs can
embed real questions. \emph{Stipulating} verbs only embed \emph{propositions} as
in \ref{ex:stipul}.
Nevertheless, both are treated in the following analysis, but with a focus on
the treatment of assertive \wh-embedding verbs.

When treating questions one will be confronted with many different
categorizations of questions. One of the first distinctions to make is the one
between \term{embedded questions} as in \ref{ex:embed} and \term{direct
questions} shown in \ref{ex:direct}.

As indicated already by the title, this paper will concern itself with the
treatment of the former and only briefly comment on the latter. The reasons for
this are numerous  %fixme I need a synonym for that
and this section is going to name the most important ones.

Early literature, such as \cite{karttunen:1977} only regards embedded
\wh-phrases, treating direct questions as semantically equivalent. To quote
Karttunen:

\ex. \a. John knows \emph{whether this sentence contains an embedded
question}.\label{ex:embed}
\b. Does this sentence pose a direct question?\label{ex:direct}


The present work will rely on \cite{gs:q} for providing the main guidelines for
the theoretic aspects and also lend from earlier works, most importantly
\cite{gs:sqpa}\footnotemark. Later on, in section \ref{sec:altq} we will also
discuss the very newest (at the time of writing) developments in question
semantics, depicted in \cite{g:is}.

\footnotetext{\cite{gs:sqpa} is actually a series of three articles
that have appeared before. Among them, we will mostly consider \cite{gs:sawhq}.
Both, the earlier paper and the later dissertation (and many other papers
originating in Amsterdam) have been kindly made
available electronically, free of charge, by the Universiteit van Amsterdam at
the following address: \url{http://dare.uva.nl}}


\chapter{Implementation} \label{sec:implementation}

Originally, \prol{} was conceptualized as a programming language for a subset of
first-order logic. This entails the predominant usage of \abbr{FOL} in section
\ref{sec:implementation}, but some higher order predicates introduced to the
language in recent years will also be put to use.

\section{Advertent Curt}

\acurt{} lends its name from the fact that it is not only aware of its own
knowledge and presuppositions, but `knows' about what individuals in its
model of the world are aware of. It is not in this sense a question
\emph{answering} system, but it may be seen as an early effort towards providing
such a system on a more sophisticated scale. While section \ref{sec:altq} shows
how direct questions can be integrated into the system with relative ease, focus
is still on providing a feasible method of implementing intensionality in what
is mostly a first-order logic computing environment.


It is also not constrained to being useful in a particular domain, like
\pn{CHAT-80} is.

\section{The Syntax of Embedded Questions}

The internal structure of the subset of interrogative sentences that will be
considered here is not a very complicated one. Since yes/no interrogatives  embed
simple propositions under a question operator only minimal modifications will
need to be made to the grammar in order to account for the latter. Constituent
interrogatives were already covered by \pn{Helpful Curt} and their syntactic
treatment does not differ in \acurt{}.

However, the syntax of embedded interrogatives is a more delicate matter. Since,
as was discussed in section \ref{sec:embed}, not every embedding verb can embed
every type of sentence (recall definition \ref{def:embed1} and definition
\ref{def:embed2}), a type system has to be established in order to account for
the different behaviour of the verbs, and said type system has to be integrated
into the process of syntactic and semantic parsing of the system.

\curt{} parses sentences of natural language using an enhanced version of \pn{Cooper
storage}\footnote{See \cite{cooper:storage2} for the original implementation.} as
presented in \cite{keller:storage}. \cite{blackburnbos:cl1} call this version
\pn{Keller Storage}. The remainder of this section will follow the nomenclature
of \cite{blackburnbos:cl1}. 

\pn{Keller Storage} is already quite capable
of dealing with more complex constituents such as nested noun phrases, but it
has its limitations. In particular, it can only handle quantifier scope
ambiguities and does not take negation into account. Moreover, it overgenerates
with some quantifier scope ambiguities\footnote{\cite{blackburnbos:cl1} mention
sentences such as \ref{ex:kellersucks}, where Keller Storage produces six
instead of the desired five readings.
\ex. One criminal knows every owner of a hash bar\label{ex:kellersucks}

Clearly, \example{a hash bar} cannot take scope over \example{every owner}, but
Keller Storage predicts this reading.}.

In order to cope with more complicated quantifier configurations, where
constraints need to be enforced, and with other kinds of ambiguities,
Blackburn and Bos present an approach utilizing underspecified representations and
dominance relations, \pn{Hole Semantics}\footnote{Also presented in % todo:cite
%Bos' paper
}. 
However none of the cases discussed so far would require the enhanced
expressive power of \pn{Hole Semantics} it was decided to not change the \curt{}
implementation's current architecture which employs \pn{Keller Storage}.

Enhancing the grammatic rules of the \curt{} system to cope with
proposition-embedding verbs is made easy by its elaborate grammatical back
end. In particular, four files are enhanced with rules to enable interpretation
of embedded questions: \code{englishGrammar.pl}, which holds generic \abbr{DCG}
rules for syntactic parsing, \code{englishLexicon.pl}, a dictionary that uses
feature structures to keep track of agreement, and two \pn{Keller Storage}
specific files, \code{semLexStorage.pl} and \code{semRulesKeller.pl}.

\begin{lstlisting}[label=whemb,caption={Syntactic and Lexical Rules for Complementizers},float]
whemb([sem:Sem]) -->
    {lexEntry(whemb,[syntax:Word,type:T])}
    , Word
    , {semLex(whemb,[sem:Sem,type:T])}
    .
lexEntry(whemb,[syntax:[that],type:propos]).
lexEntry(whemb,[syntax:[whether],type:question]).
\end{lstlisting}

The grammatic rules are extended by the introduction of three new categories,
\code{whemb}, \code{tbar} and \code{ivtbar}.

The \code{whemb} and \code{ivtbar} categories have a \code{type} parameter, that dictates
their behaviour, although their respective type systems never cooperate with
each other. The type of the \code{whemb} category can hold the values
\code{propos}, for a proposition embedding complementizer, like \example{that},
and \code{question}, for a complementizer that turns a proposition into a yes/no
interrogative, like \example{if} and \example{whether} (listing \ref{whemb}).


\begin{lstlisting}[label=whembsemLex,caption={Semantic Rules for Complementizers},float]
semLex(whemb,M) :-
    M = [sem:[lam(S,S)],type:propos]
    .
semLex(whemb,M) :-
    M = [sem:[lam(Q, que([],[],Q))],type:question]
    .
\end{lstlisting}

The types now determine whether to produce an embedded yes/no
interrogative or just pass through as a simple proposition (in which case a
simple $\lambda S. S$ serves as an identity function), and the
\code{semLex/2} predicate takes care of this (listing \ref{whembsemLex}). In the
latter case, a new sort of question is introduced, namely one that does not
quantify over a free variable. It has the form \lstinline!que([],[],Q)! and is
representative of $?.\varphi$.

The \code{tbar} rule has three different right hand sides. 
In one of them, it first parses a \code{whemb} and combines it with a sentential
category \code{s} (listing \ref{tbar}, line 1). The second case omits the
complementizer and just passes the sentence as a normal proposition (line 7).
Next, the final rule takes care of embedding questions (line 12).

In every case, the sentence embedded is first fully reduced by the predicates
shown in listing \ref{tbarCombine}. This has to be done in order to avoid
unsaturated formulae in the end result. Thus the $\bar{t}$ category can   produce
one of the following: $\varphi$, $?.\varphi$ or $?x.\varphi$.

\begin{lstlisting}[label=tbar,caption={Semantic Rules for $\bar{t}$},float]
tbar([sem:TBar]) -->
    whemb([sem:WHemb])
    , s([coord:no,sem:S])
    , { combine(tbar:TBar,[whemb:WHemb,s:S]) } .

tbar([sem:TBar]) -->
    s([coord:no,sem:S])
    , { combine(tbar:TBar,[s:S]) } .
 
tbar([sem:TBar]) -->
    q([sem:Q])
    , { combine(tbar:TBar,[s:Q]) } .
\end{lstlisting}

\begin{lstlisting}[label=tbarCombine,caption={$\bar{t}$ is responsible for beta-converting the embedded sentence.},float]
combine(tbar:[app(WH,SRed)],[whemb:[WH],s:S]) :-
    betaConvert(S,SRed).

combine(tbar:[SRed],[s:S]) :-
    betaConvert(S,SRed).
\end{lstlisting}

As a final step, \code{ivtbar} (or $IV/\bar{t})$ in a formal type system), takes
the result of \code{tbar} and embeds it under a suitable verb, taking the verb
type (one of \code{factive}, \code{resolutive}, \code{question} and
\code{propos}) and the nature of the embedded sentence into account. Listing
\ref{ivtbar} shows only one case, for \code{question} embedding verbs, and the
(semantic) lexical rules have been omitted. See the
appendix %todo: section number
for the full definition.

\begin{lstlisting}[label=ivtbar,caption={A question embedding VP requires its antecedent to be a question.},float]
vp([coord:no,inf:Inf,num:Num,gap:[],sem:VP]) -->
    ivtbar([inf:Inf,num:Num,sem:IVTBAR,type:question])
   , tbar([sem:TBAR])
   , { TBAR = [app(lam(X,que(_,_,X)),X)]
       , combine(vp:VP,[ivtbar:IVTBAR,tbar:TBAR])
     }.
\end{lstlisting}

\section{Modality and Possible Worlds in \prol{}}
\label{sec:indices}

\section{Interpretation}

\chapter{Discussion}\label{sec:discussion}

We have seen, the \curt{} system is quite capable of treating questions this way.
Having implemented modality and possible worlds in such a fashion, let's go on
and implement some even cooler stuff (tense, aspect, negation, comparatives,
anaphora). Basing all this on a nice implementation of DRT (discussed in
\cite{kampreyle:drt}) like the one given in \cite{blackburnbos:cl2} might be
interesting. The Pronto \abbr{QA} system of \cite{prontoqa} already defines a
means of representing questions and queries against a Discourse Representation
Structure.

Burhans and Shapiro

Nelken and Francesz, give something that might not need sets of propositions

\appendix

\chapter{Appendix: The Source Code}

\section{\code{advertentCurt.pl}}


\chapter{Appendix: Examples}

\subsection{An Example Model} \label{sec:examplemodel}

\begin{eqnarray*}
  M &=& \set{\pair{1,\pair{D_1,F_1,\varphi_1}},\pair{2,D_2,F_2,\varphi_2}} \\
  D_1 &=& \set{d_1,d_2,d_3} \\
  F_1 &=& \{f(0,mia,d_1), f(0,vincent,d_2), f(0,jody,d_3),\\
  && f(1,woman,\set{ d_1,d_3 }), f(1,man,\set{ d_2 }),
  f(2,like,\set{\pair{d_2,d_1},\pair{d_3,d_2}})\} \\
  \phi &=& \exists x,y,z [mia(x)\wedge woman(x)\wedge \wedge vincent(y)\wedge
  man(y)\wedge jody(z)\wedge woman(z) \\&&\wedge x \neq y \wedge x\neq z \wedge z
  \neq y\wedge like(x)(y)\wedge like(y)(z)]\\
  D_2 &=& 
\end{eqnarray*}

\newpage

\bibliography{aleksbib}

\end{document}

