\documentclass[12pt,a4paper]{article}

%\usepackage[english]{babel}

\usepackage{amssymb,qtree,natbib,aleks,linguex}

\usepackage{amsthm}

%% note that this file depends on being processed with XeTeX
\usepackage{fontspec}
\usepackage{xltxtra,xunicode}

\usepackage{mathpazo}
% font specifications (don't overuse ;-)
\defaultfontfeatures{Mapping=tex-text} % converts LaTeX specials (``quotes'' --- dashes etc.) to unicode
\setromanfont [Ligatures={Common,Rare}, BoldFont={Fontin Bold}, ItalicFont={Fontin Italic}]{Fontin Regular}
\setsansfont [Ligatures={Common}, BoldFont={Fontin Sans Bold}, ItalicFont={Fontin Sans Italic}]{Fontin Sans}


\newcommand{\abbr}{\textsf} % how to render abbreviations like IL or Ty2
\newcommand{\stress}{\textbf} % put stress somewhere (as opposed to \emph)
\newcommand{\term}{\textsf} % introduce a term
\newcommand{\code}{\texttt} % layout of in-line code snippets
\newcommand{\pn}{\textsf} % how to render proper names like Shrdlu
\newcommand{\url}[1]{\code{http://#1}} % how urls appear
\newcommand{\example}{\textit} % in-text examples

% commonly used phrases
\newcommand{\Disc}{\ensuremath{\mathcal{D}}} % the discourse
\newcommand{\wh}{\textsc{wh}} %

%names
\newcommand{\curt}{\pn{Curt}\mbox{ }}
\newcommand{\acurt}{\pn{Advertent Curt}\mbox{ }}
\newcommand{\prol}{\pn{Prolog}\mbox{ }}

%theorem environments
\theoremstyle{remark} \newtheorem*{termin}{Definition} % terminology. Could use a
%better title fixme

\bibliographystyle{plainnat}
\bibpunct{(}{)}{;}{a}{,}{,}

\author{Aleksandar Lyubenov Dimitrov\\
$\langle$\href{mailto:aleks.dimitrov@googlemail.com}{aleks.dimitrov@googlemail.com}$\rangle$}
\title{From Questions to Queries – Enhancing the Curt System by a Theory and Implementation of Embedded WH-Questions}

\usepackage{fontspec}
\usepackage[xetex,colorlinks,citecolor=magenta]{hyperref}
\begin{document}

\maketitle
\tableofcontents
\newpage

\section*{Introduction}


Growing efforts have thus been made to provide a technology that is (going to
be) able to extract logical formulae from sentences of ordinary language and
establish relations between objects of its logical representation. In addition,
providing a means to compute with said representations is will eventually enable
the machine-processing of meaning.  In doing so, the ultimate goal may be to
enable a man-made system to \emph{understand} what a humans communicate.

Exploration of this  interesting field of linguistics has shown that
the notion of \emph{understanding} is far more than a linguistic concern, but
involves at least findings in the domain of logics, (cognitive) psychology and
probably even philosophy.

Development of exhaustive systems in this area appears to be a daunting task of
quite some scope. Therefore most approaches have been centered on providing
coverage of at least a fragment of natural language. Many such systems exist and
they prove to be increasingly successful. Some of them will be discussed here
and presenting an extension to one system in particular is the main matter of
this Bachelor's thesis.

In doing so, the present work will aim to provide the \curt system of
\cite{blackburnbos:cl1} with the mechanics of dealing with more than one
partition of its logical domain, viz. the ability to handle more than one
\term{possible} or \term{accessible world}. As a ground for development of this
work, the treatment of \term{embedded questions} was chosen as one that
naturally demands a possibility to account for more than one division in a
system's \emph{understanding} or representation of a universe of objects and
relations. 

As 

The first section gives a conceptual overview, introducing the \curt
system and comparing it to other similar systems in existence. It then goes on
to explain the target domain of natural language the implementation has to
account for.
The next section concerns itself with a more detailed analysis of the theoretic
assumptions of this implementation and justifies some choices and restrictions
that had to be made because of the limited scope of this analysis. While the
third section will display and clarify the actual implementation on top of the
\curt system, the last section will discuss the results of this work and also
give a prospect on possible further work on the subject.

\subsection*{Acknowledgements}

Frank... who else?

\section{Concepts}

The basis of 

%fixme: to be formalized
Where we use ``natural language'', usually the English language will be treated
in practice, though it would be nice to come up with treatments of other
languages. We abstract away from pragmatic affairs and concern ourselves with
mostly the purely semantic/logic domain, so this should be feasible to achieve.

A brief overview of the premisses for the theory (which itself lays ground to
the implementation).

\subsection{The Curt System} \label{sec:curt}

\pn{Curt}, a system written in \prol that is able to interpret basic utterances
in natural language and store them in expressions of first order logic, is
presented in \cite[chapter 6]{blackburnbos:cl1}. While relatively limited when
compared to other systems presented in \ref{sec:comparision} in terms of lexical
and grammatical coverage, it provides a sound basis for extended work because of
its coherent design and relative simplicity.

Besides parsing and translating utterances into logic, \curt is able to perform
basic checks for \emph{consistency} and \emph{informativity} on the resulting
formulae -- within the bounds of the tools it uses.

% FIXME: don't really like this footnote
\footnotetext{Here, the notion of \emph{consistency} is meant as \emph{validity}
of the input and \emph{informativity} denotes \emph{validity}. Thus, if
\Disc, the discourse so far and $\phi$ a new input token (fully parsed
sentence), then consistency means $\Disc \to \phi$ holds and informativity
means $\Disc \nvDash \phi$.}


It can thus detect incoherent discourse and would reject a sequence of
sentences such as \emph{``Mary likes every man. Peter is a man. Mary does not
like Peter''}, claiming it found a contradiction in its input. For every input
token, \pn{Curt} will also try to infer whether the current discourse would
entail the information given in its newest token and is thus irrelevant with
regard to previous utterances – effectively allowing it to
reduce the size of its internal representation of the discourse by not adding
clauses to it which would not contribute to the overall informational value
already present in the program.

The system is introduced during the final chapter of \cite{blackburnbos:cl1},
incrementally increasing in features and complexity. Support for external
inferencing via theorem provers and model builders is added in a step-by-step
fashion and finally an ontology and a straight-forward way of parsing and
answering direct questions for noun phrases conclude its development. The theory
and implementation of this paper is meant as a replacement and enhancement for
the latter feature. This will also happen in an incremental manner and new
features will be introduced one at a time.

\subsection{General Comparison of the Curt System}\label{sec:comparision}

This section will briefly discuss how the features of the Curt system compare to
other similar systems or systems with the same goal, namely implementing a
system that is able to model a logic representation of human language.

\subsubsection{Early Attempts: \pn{Shrdlu} and \pn{Cyc}}
%TODO: need some reference for this section!

Interest in deep semantic parsing of natural language has existed since the very
beginning of the informational age. One of the first systems to achieve public
recognition, the \pn{Shrdlu} system 

While the \pn{Curt} system is unique in its approach to deep semantic analysis, its
architecture has been crafted in a pragmatic manner. The authors chose to use
external tools where it was possible in order to avoid reimplementation of
existing functionality and be able to focus on novel parts of the
implementation\footnote{The name of the \pn{Curt} system reflects this choice.
It stands for \emph{\textbf{C}lever \textbf{u}se of \textbf{r}easoning
\textbf{t}ools}.}. This also allows for great flexibility in adopting new
solutions and improvements made in the field of inferencing tools for first
order predicate logic.

\subsection{Extending the Curt System}

At the end of \cite{blackburnbos:cl1}, \curt is able to cope with a small
grammar and provides some features for interacting with humans in natural
language. It does only understand basic extensional transitive verbs - modal
verbs or the treatment of questions are not implemented. Anaphora resolution
still remains an open issue. The grammar is rather limited and \curt only comes
with a basic vocabulary oriented around the setting of a popular movie.

The aim of this paper is to introduce the reader into question semantics as it
will be relevant for implementing a basic capability of treating intensional
verbs like `to \stress{know} whether something is the case' and `to
\stress{believe} that something might be the case' and then present such an
implementation within the framework of the \curt system. Further and more
advanced investigations will be dealt with at the end of this discussion or left
for further research.

\section{Theory}

This section first gives an overview over the terminology that will be used for
the discussion of the formal treatment of questions. It then elaborates on the
differences between different logical formalisms employed in the treatment of
natural language questions. After a small overview on the types of questions
(and pointing out the subset of questions that will be treated), %fixme maybe a
% bit sloppy
the discussion
turns to how those questions are best treated in the given formalism.

The reader should note that all choices presented in this section have been made
with the implementation of section \label{sec:implementation} in mind.
% todo talk a bit more


\subsection{Terminology}

Following \cite{gs:q}, we will not use the natural natural language word
``question'' but distinguish between \term{questions},
\term{interrogative sentences} and \term{interrogative acts} in the following
way:

\begin{termin}
  An \term{interrogative} is a certain type of sentence in natural language
  distinguished by certain features\footnote{To quote \cite{gs:q}: \quote{[An
  interrogative sentence is] characterized by word order, intonation, question
  mark, the occurrence of interrogative pronouns}.}
  whereas an \term{interrogative act} refers to the utterance of an
  \emph{interrogative}. Finally, a 
  \term{question} will denote the semantic entity uttered by the
  \emph{interrogative act}, whichever form it may have.
\end{termin}

\abbr{Ty2} is two-sorted type logic\footnote{See \cite{gallin:ty2} for a
discussion and \cite{gs:sqpa} for its usage in describing the semantics of
questions}. \abbr{IL} is intensional logic\footnote{As used by \cite{ptq}}. 
An embedded \wh-phrase looks
like:%fixme

\ex. John \stress{knows} \emph{whether} good examples are hard to come up with.

After having elaborated on the grounds of the implementation, we will now
display the implementation's underlying concepts. % FIXME this sentence is too short

%OUTLINE Presenting a conceptual overview, while arguing for and against stuff.

First we will present a brief overview of the considerations regarding the
choice of an appropriate formalism for the system in section \ref{sec:formal}
to then compare it with already existing implementations of question answering
systems in \ref{sec:othercrap}.

\subsection{Formalism}\label{sec:formal}

% FIXME if we have a `Terminology' section, all those `henceforths' aren't really
% necessary anymore.

Most of the traditional literature in natural language semantics follows
\cite{ptq} and thus also makes use of its underlying formalism,
\emph{Intensional Logic} (henceforth \abbr{IL}). \cite{gs:sqpa} (and many subsequent
publications) however also use two-sorted type logic (henceforth \abbr{Ty2}) as
proposed in \cite{gallin:ty2}. \cite{z:ilty2} discusses the differences and
similarities in expressive power between \abbr{Ty2} and \abbr{IL} and comes to
the conclusion that the set of expressions contained in the former but not in
the latter might not be relevant for the treatment of natural language
semantics. Although he proposes a translation scheme between both languages,
translation from \abbr{Ty2} to \abbr{IL} seems to sometimes produce very complex
\abbr{IL} terms and is therefore avoided here. In fact, \abbr{IL} will be only
marginally relevant for the presentation of \acurt – only the type system
presented in \cite{ptq} is carried over.

% OK till now, but the rest is a fixme :-(

Since this paper is not concerned with possible extensions of the underlying
theories of treating natural language questions, but rather with their
implementation in the programming language \pn{Prolog} several additional
constraints have to be considered. Among them the \emph{feasibility} of any
possible approach takes highest precedence.

To quote \cite{blackburnbos:cl1}:

\begin{quote}
  Questions, unlike assertions, don't have truth-values, so it would be
  misleading to represent them as ordinary formulas.
\end{quote}

It is important to note that this claim is not undisputed and
\cite{karttunen:1977}, argues against it. The present approach, however will
base itself on this assumption and extend it by more formal approaches than the
one given by Blackburn \& Bos. Specifically, the work by J. Groenendijk and M. Stokhof
will lay ground to the implementation. But in order to realise their formal
discussions computationally, modifications will have to be made, and even
compromises.

As was already mentioned in \ref{sec:curt}, Curt is based on an
implementation and the use of external tools only capable of treating first
order logic – an extensional framework. \prol on the other hand, is a language
that can very well make use of higher order formalisms for
calculation\footnote{The two most popular predicates in \prol used for
implementing higher-order logic are
\code{call/n} and \code{apply/3}. This paper is based on \pn{SWI-Prolog} which
provides an implementation of the former, but not of the latter. See
\cite{naish:prolhio} for a discussion of higher order functional programming in
\prol.}.

While there exist inferencing tools for modal logic\footnote{See %todo todo and todo 
for discussions. The interested reader is also referred to \pn{Molle}
(\url{http://molle.sf.net}), a modal theorem prover that comes with a graphical user
interface which allows interactive investigation of the constructed indices.},
a restructuring of the \pn{Curt} system to take advantage of them would be
beyond the scope of this thesis.

It was therefore decided to avoid modal logic and implement a pragmatic approach
to representing the indices of higher-order logic in \prol in section
\ref{sec:indices}. This logic however has to be augmented with question
operators as already used by \cite{karttunen:1977}. % todo: writeme!

\cite{g:is} presents an approach to implementing inquisitive semantics in
predicate logic.

\subsection{On the Proper Treatment of Questions}

Question semantics are an interesting subproblem of modern research on natural
language semantics. While traditional approaches based on \abbr{IL} and newer
accounts also employ plain extensional logic, % todo: which ones 
the treatment
of questions seems to demand that we introduce a new kind of logic formalism.
But as interesting topics go, the semantics of questions has attracted many
different views on this issue.

Often question semantics has been reduced to a problem of
pragmatics, treating direct interrogatives as nothing more than a
paraphrase of embedded interrogatives.\cite{tichy}
even goes as far as claiming equality in semantic (truth) value between
indicatives and interrogatives. The benefit of such an approach would be the
strict reliance on traditional analyses, where interrogatives would only infer a
different `attitude' in the speaker's utterance. Putting this attitude into
context and making it yield the correct results would then be left to
pragmatics.

This approach has been challenged and proven unfeasible by \cite{gs:q},
who have also proposed interesting extensions
to the existing logic frameworks used in natural language semantics.
More recent publications, such as \cite{gal:tmbi}
address this topic and aim to introduce notions of \term{information change
potential} as opposed to the traditional focus of natural language semantics on
\term{truth values}.

The present work will rely on \cite{gs:q} for providing the main guidelines for
the theoretic aspects and also lend from earlier works, most importantly
\cite{gs:sqpa}\footnotemark. Later on, in section \ref{sec:altq} we will also
discuss the very newest (at the time of writing) developments in question
semantics, depicted in \cite{g:is}.

\footnotetext{\cite{gs:sqpa} is actually a series of three articles
that have appeared before. Among them, we will mostly consider \cite{gs:sawhq}.
Both, the earlier paper and the later dissertation (and many other papers
originating in Amsterdam) have been kindly made
available electronically, free of charge, by the Universiteit van Amsterdam at
the following address: \url{http://dare.uva.nl}}


\subsubsection{Pragmatics and Semantics of Questions}

In choosing to treat only embedded questions, we can escape most pragmatic
considerations and focus on the (implementation of) pure semantics of
interrogatives.

\subsection{Classification of Interrogatives}

When treating questions one will be confronted with many different
categorizations of questions. One of the first distinctions to make is the one
between \term{embedded questions} as in \ref{ex:embed} and \term{direct
questions} shown in \ref{ex:direct}.

\ex. \a. John knows \emph{whether this sentence contains an embedded
question}.\label{ex:embed}
\b. Does this sentence pose a direct question?\label{ex:direct}

As indicated already by the title, this paper will concern itself with the
treatment of the former and only briefly comment on the latter. The reasons for
this are numerous  %fixme I need a synonym for that
and this section is going to name the most important ones.

Early literature, such as \cite{karttunen:1977} only regards embedded
\wh-phrases, treating direct questions as semantically equivalent. To quote
Karttunen:

\begin{quote}A direct question can be treated as semantically equivalent to a
certain kind of declarative sentence containing the corresponding indirect
question embedded under  a suitable `performative' verb.\end{quote}

Using this strategy, Karttunen escapes the treatment of direct  questions and
thus gives an approach for indirect questions only.

\subsubsection{An Ontology of Embedding Verbs}

\cite{karttunen:1977} introduces means of classification for \wh-embedding
verbs, the nature of which \cite{lahiri:diss} covers in a much broader scope.
The different verb classes ask for different strategies of treating the
\wh-clauses they embed and also pose different restrictions on the type of
clause they expect.

The present work will not consider a full analysis of \wh-embedding verbs, but
confine itself on a few ones, namely \example{know}, \example{think} and
\example{believe}. The grammar implementation of \acurt allows for \wh-embedding
verbs to have a \code{type} field, determining their semantic denotation. The two
types used so far are \code{assert} for assertive verbs such as \example{know}
and \code{stipul} for stipulating verbs, such as \example{believe} and
\example{think}.

\subsubsection{Grammatic Constraints}

With this ontology in mind, the next step is to analyse and enumerate the sorts
of grammatic constructions that are to be treated. The following examples will
illustrate them.

\begin{itemize}
  \item Alternative Questions
  \ex. \a. Mia \emph{knows} whether Vincent snorts.
  \b. * Mia \emph{believes} whether Vincent snorts.
  \b. * Mia \emph{thinks} whether Vincent snorts.
  
  \item Enumerative Questions
  \ex. \a.  Mia \emph{knows} who snorts.
  \b. * Mia \emph{believes} who snorts.
  \b. * Mia \emph{thinks} who snorts.

  \item Embedding Propositions
  \ex. \label{ex:stipul}
  \a. Mia \emph{believes} Vincent snorts.
  \b. * Mia \emph{believes} who snorts.

\end{itemize}

An important observation to be made here is that only \emph{assertive} verbs can
embed real questions. \emph{Stipulating} verbs only embed \emph{propositions} as
in \ref{ex:stipul}.
Nevertheless, both are treated in the following analysis, but with a focus on
the treatment of assertive \wh-embedding verbs.


\subsubsection{Extensional WH-embedding Verbs}

Like \emph{know}.

We first derive know(john,all(X,imp(woman(X),love(X,john)))) from the syntax of
john knows whether every woman loves john, then we see if
all(X,imp(woman(X),love(X,john))) is valid (uninformative) and ridicule John, if
it's not valid and not consistent, assert his knowledge of the fact that it's
not true. If it's true, we assert he knows that.

For whether we can ridicule him if it's not true (inconsistent).
The MB/TP do a comparison \Disc to all(X,imp(woman(X),love(X,john)))

\subsubsection{Intensional WH-embedding Verbs}

Like \emph{believe}.

\section{Implementation}

\subsection{Existing Implementations of Similar Systems}\label{sec:othercrap}

Enhancing \curt by an implementation dealing with questions in natural language
sparks interest in whether there have been similar efforts made in the past. No
further extensions to the \curt system aiming at comparable goals are known to the
author at the time of writing. However, there is a lively interest in providing
so-called question answering or QA-systems since the early stages of computer
science. % fixme; cite some ancient 60's paper
This section discusses a few of the more elaborate approaches: first an early
but already quite impressive implementation is shown, followed by a newer approach
using embodied agents and finally an overview over the many database-based
QA-systems is given. Finally, all this is compared what the current extension to
\curt can do.


\subsubsection{Godot, the Talking Robot}

\subsubsection{CHAT-80}

\subsubsection{Modern Intensional Database Approaches}

\subsubsection{Advertent Curt}

\acurt lends its name from the fact that it is not only aware of its own
knowledge and presuppositions, but `knows' about what individuals in its
model of the world are aware of. It is not in this sense a question
\emph{answering} system, but it may be seen as an early effort towards providing
such a system on a more sophisticated scale. While section \ref{sec:altq} shows
how direct questions can be integrated into the system with relative ease, focus
is still on providing a feasible method of implementing intensionality in what
is mostly a first-order logic computing environment.


It is also not constrained to being useful in a particular domain, like
\pn{CHAT-80} is.

\subsection{Modality and Possible Worlds}
\label{sec:indices}

We have seen that some embedding verbs may not be interpreted without
implementing a higher order formalism that is able to model the notion of
\emph{possible} and \emph{accessible} worlds. While foobar %fixme! Nelken &
% Francez 2001
argue for an extensional treatment of questions, they already % n & f 2006
retract their claims and admit that a ``minimal account of intensionality'' has
to be assumed. While they propose a two-world system later adopted in
\cite{g:is}, this is not going to suffice if we want to model each individual's
knowledge in order to be able to discriminate people's knowledge and beliefs.
%fixme bla bla

The underlying theory for 

\subsubsection{Representing Intensionality in \prol}

\prol programs can be self-modifying, or dynamic. This means that the program is
able to change its own state and thus store information. 

\subsubsection{Disjunction and Ambiguity}
Consider the following sentences:

\ex. \label{ex:disjunction} \a. Mia or Jody dances
\b. Either Butch snores or Vincent is using the chainsaw

\pn{Helpful Curt}, with its unpartitioned logical space and its lack of
the awareness of more than one (possible) world, would generate only one model
to represent the facts of \ref{ex:disjunction}. It is therefore only able to
generate \emph{one} possible answer to both questions, effectively accounting only
for one possible constituent of the disjunction, prefixing it with a `maybe' to
indicate its uncertainty.

Another example where \curt has to sacrifice exhaustiveness over its one-world
representation of the world is given in \ref{ex:ambig}

\ex.\label{ex:ambig} Every boxer likes a woman.

This will generally generate two readings, but only one model -- with only one
world. And further discourse will have to choose between one of the readings and
stick with the minimal model constructed by the model builder. This is not
really satisfactory and extending the dimensions of knowledge within \curt seems
necessary in order to treat such examples.

The previous work on \acurt enables it to generate a \term{choice point} -- and
moreover, not pursue only one path of it, but both at the same time, by
generating more than one world in its model.
Unfortunately, this will build up very large models over time and computations
with it might prove to be very complex and resource demanding. Adding such a
functionality to the system, however, is not a hard task and hopefully the
technique employed here can be further refined. %todo: maybe add a footnote
%about future prospects or so

\subsubsection{Defining the Grammar Rules}

\curt parses sentences of natural language using an enhanced version of \pn{Cooper
storage}\footnote{See \cite{cooper:storage2} for the original implementation.} as
presented in \cite{keller:storage}. \cite{blackburnbos:cl1} call this version
\pn{Keller storage}. That name will be used from here % FIXME sucks. Please change
to refer to this particular implementation as presented in
\cite{blackburnbos:cl1}. 

\pn{Keller storage} is already quite capable
of dealing with more complex constituents such as nested noun phrases, but it
has its limitations. In particular, it can only handle quantifier scope
ambiguities and does not take negation into account. Moreover, it overgenerates
with some quantifier scope ambiguities\footnote{\cite{blackburnbos:cl1} mention
sentences such as \ref{ex:kellersucks}, where Keller storage produces six
instead of the desired five readings.
\ex. One criminal knows every owner of a hash bar\label{ex:kellersucks}

Clearly, \example{a hash bar} cannot take scope over \example{every owner}, but
Keller storage predicts this reading.}.

In order to cope with more complicated quantifier configurations, where
constraints need to be enforced, and with other kinds of ambiguities,
Blackburn and Bos present an approach utilizing underspecified representations and
dominance relations, \pn{Hole Semantics}\footnote{Also presented in % todo:cite
%Bos' paper
}. Since, however, none of the cases discussed so far would require the enhanced
expressive power of \pn{Hole Semantics} it was decided to go on using \pn{Keller
Storage} instead, since this is what the original implementation of the \curt
system relies on.
%todo: fix capitalization in Keller storage all over the document

Enhancing the grammatic rules of the \curt system to cope with
proposition-embedding verbs is made trivial by its elaborate grammatical back
end. \ref{ex:kel1} is an example of the logic behind the grammar that is put to
use.

\ex. \label{ex:kel1} Mia knows whether Vincent snorts.\\
\Tree
[.$t$ [.$T$ { $\lambda X.X(Mia)$\\Mia } ]
[.$IV$ [.{$IV/\bar{t}$} { $\lambda X.know(X)$\\knows } ]
[.{$\bar{t}$} [.$\bar{t}/t$ whether ] [.{ $t$\\$snort(Vincent)$ } 
[.$T$ { $\lambda X.X(Vincent)$\\Vincent } ]
[.$IV$ { $\lambda X.snort(X)$\\snorts } ] ] ] ] ]

\footnotetext{Note that \ref{ex:kel1} is shown using the same syntactic type system used in
\cite{gs:sawhq}, which extends the system given in \cite{ptq} by $\bar{t}$ to
denote embedded \wh-constituents.}

\subsection{Treatment of Baker Ambiguities}

\subsection{Alternative Questions – Inquisitive Semantics}\label{sec:altq}

\cite{g:is} introduces a novel approach to treating questions in natural
semantics: \pn{Inquisitive Semantics}. He presents a new logic
formalism aimed at providing a wider coverage of the intentions of question in
human language. The main source of the ``inquisitiveness'' of statements is
given by the interpretation of \emph{disjunction}. We will now show how to
incorporate this novel approach into the \pn{Curt} system.

We will, however depart from the path we have been following throughout this
paper and first implement a theory for direct questions. Only then will we extend
it to also cover embedded alternative questions.

\section{Discussion}

We have seen, the Curt system is quite capable of treating questions this way.
Having implemented modality and possible worlds in such a fashion, let's go on
and implement some even cooler stuff (tense, aspect, negation, comparatives,
anaphora). Basing all this on a nice implementation of DRT (discussed in
\cite{kampreyle:drt}) like the one given in \cite{blackburnbos:cl2} might be
interesting.

\section{Appendix}

\subsection{The Source Code}

\subsection{Whatnot}

\bibliography{aleksbib}

\end{document}

