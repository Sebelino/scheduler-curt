\documentclass{beamer}

\mode<beamer>{
\usetheme{Singapore}
%Berlin
%\usecolortheme{rose}
}

\usepackage[english]{babel}
\usepackage{qtree}
\usepackage{ucs}
\usepackage[utf8x]{inputenc}
\usepackage{amsfonts,amsmath}
\usepackage{linguex}
\usepackage{color}

\newcommand{\stress}{\textbf}
\title{From Questions to Queries: Asking Prolog the Natural Way}
\institute{Universität Tübingen}
\author{Aleksandar L. Dimitrov}

\begin{document}
\begin{frame}
  \titlepage
\end{frame}

\begin{frame}
  \frametitle{Outline}
  \tableofcontents
\end{frame}

\section{What We Have So Far}

\subsection{The Curt System}
\begin{frame}
\frametitle{The Curt System so Far}
  \begin{itemize}
    \item[$\to$] Newest version: Sensitive Curt
    \item \stress{Features:}
    \begin{itemize}
      \item Uses external tools to provide enhanced feature set
      \item Can thus \stress{detect inconsistent additions} to existing discourse
      \item and \stress{detect uninformative additions} to existing discourse
    \end{itemize}
    \item \stress{Shortcomings:}
    \begin{itemize}
      \item Does not account for $\alpha$-equivalent readings
      \item Does not know about worlds and situations
      \item Has no background knowledge apart from the discourse
      \item Is hard to interface with as a normal human
    \end{itemize}
  \end{itemize}
\end{frame}

\section{Blackburn \& Bos: Extending Curt}

\subsection{Scrupulous Curt}
\begin{frame}
\frametitle{Scrupulous Curt}
  \begin{itemize}
    \item Recognizes ``a man likes a hash bar'' is not ambiguous
    \item \ldots by proving ``$\leftrightarrow$'' between the two readings
    \item Only disambiguates at request
    \item Does not produce models after disambiguation
    \item Is unable to construct a model where one holds and the other doesn't
    \item \emph{Failure to prove something does not prove the non-existance of a proof}
  \end{itemize}
\end{frame}

\subsection{Knowledgeable Curt}
\begin{frame}
\frametitle{Knowledgeable Curt: Know What's Going on in the Human World}
  \begin{itemize}
    \item Knowledgeable Curt introduces \stress{background knowledge}
    \item Features
    \begin{itemize}
      \item Distinguishes between \emph{plants} and \emph{humans}
      \item Adds new types of knowledge to our discourse: \stress{it takes knowledge}
      \item[$\to$] from \pause the lexicon\pause, the world\pause{} and the situational
      context
    \end{itemize}
    \item Our new discourse:
    \begin{equation}
    LK \wedge WK \wedge SK \wedge DK
    \end{equation}
  \end{itemize}
\end{frame}

\begin{frame}
\frametitle{Lexical Knowledge}
  Keywords: \emph{Lexical Semantics}, \emph{ontology}
  \begin{itemize}
    \item Knowledgeable Curt introduces \emph{Ontologies}
    \item \Tree [.Organism Animal Plant [.Person Man Woman ] ]
    \item Ontologies can be \emph{subject of dispute}
    \item \texttt{lexicalKnowledge.pl} uses \alert{triggers} to enhance
      efficiency.
  \end{itemize}
\end{frame}

\begin{frame}
\frametitle{Lexical Knowledge, Cntd.}
  \begin{itemize}
    \item A particular ontology can only be applicable to a certain domain
    \item Ontologies can help distinguish \stress{homonyms}
    \item[$\to$] $\forall X [\texttt{grunt}(X) \wedge \texttt{person}(X)]$ vs.  $\forall X [ \texttt{grunt}(X) \wedge \texttt{utterance}(X) ]$
    vs. $\forall X [\texttt{grunt}(X) \wedge \texttt{fish}(X)]$
  \end{itemize}
\end{frame}

\begin{frame}
\frametitle{World Knowledge}
  \begin{itemize}
    \item Being a \texttt{wife} implies being \texttt{married}
    \item Distinction between \stress{world knowledge} and \stress{lexical
    knowledge} can be fuzzy:
    \begin{itemize}
      \item being a \texttt{student} implies being an \texttt{undergraduate}.
      \item \ldots{} but couldn't a \texttt{student} be a child of \texttt{undergraduate}?
    \end{itemize}
    \item \stress{Inheritance relationships:} no thing is a subpart of itself, no thing is a subpart of two distinct things.
    \item \stress{Really?} No thing is owned by itself? No thing is owned by two
    distinct things?
    \item[$\to$] Distinguish between notions of \emph{have}.
  \end{itemize}
\end{frame}

\begin{frame}
\frametitle{Situational Knowledge}
  \begin{itemize}
    \item Discourse knowledge
    \begin{itemize}
      \item When we talk about two different things, we don't talk about the
	same thing.
    \end{itemize}
    \item All three types of knowledge can interact in interesting
      ways\ldots{}\pause
    \item consider: \emph{every car has a radio}.
  \end{itemize}
\end{frame}

\subsection{Helpful Curt}
\begin{frame}
\frametitle{Helpful Curt: Asking the Natural Way}
  \begin{itemize}
    \item For now, \texttt{readings/0} and \texttt{models/0} were the only way of
    querying Curt
    \item How about natural language questions?
    \begin{enumerate}
      \item Is Mia married?
      \item Who owns a hash bar?
      \item Does every man like a gun?
      \item Who did not shoot Marvin?
    \end{enumerate}
    \item Building models where possible answers hold
  \end{itemize}
\end{frame}

\begin{frame}
\frametitle{Helping Curt Help}
  \begin{itemize}
    \item Introducing the \texttt{que/3} predicate.
    \begin{itemize}
      \item basically a \emph{variable binder}:
      \item[$\to$] $que(X, person(X), not(shot(X. marvin))).$
      \item[$\to$] $que(X, customer(X), some(Y, and(fdshake(Y), order(X,Y)))).$
    \end{itemize}
    \item No first order terms or formulas
    \item[$\to$] Translate:
    \begin{equation}
      que(X,R,N) \to some(X, and (R,N))
    \end{equation}
    \item[$\to$] Now we know whether the question makes sense - and if we can
    answer it.
    \item Next: free the variable and unify it with the model: $satisfy(X, and(R,N))$
  \end{itemize}
\end{frame}

\section{Excursion: Question Semantics}
\begin{frame}
\frametitle{Question Semantics}
  Two main theories:
  \begin{enumerate}
    \item<1-| alert@3> Lauri Karttunen
    \item<2-> Groenendijk \& Stokhof
  \end{enumerate}
\end{frame}

\subsection{Karttunen 77}
\begin{frame}
  \frametitle{Karttunen 77: Types of Questions}
  \begin{itemize}
    \item<1-> Two (syntactic) types of questions:
      \begin{enumerate}
	\item<2-|alert@4> Direct questions
	\item<3-|alert@5> Indirect questions
      \end{enumerate}
    \item<4-> \emph{Which book did Mary read?}, \emph{Who is John meeting?}
    \item<5-> I ask\ldots{} \emph{\dots{} whether it is raining},
      \emph{\dots{} who John met}
    \item<6-> Karttunen treats only \alert{indirect questions}
  \end{itemize}
\end{frame}

\begin{frame}
  \frametitle{Karttunen 77: Types of Questions, contd.}
  \begin{itemize}
    \item<1-> Two (semantic) types of questions:
      \begin{enumerate}
	\item<2-|alert@4> Alternative questions \only<5->{\stress<5>{Yes/No questions}}
	\item<3-|alert@6> Wh-questions
      \end{enumerate}
    \item<4-> \emph{Does Mary smoke?}, \emph{Will you come to the party?}
    \item<6-> \emph{Who owns a hash bar?}, \emph{Who is Mia married to?}
    \item<7-> Karttunen considers \stress{alternative questions} syntactically
      degenerate variants of \stress{Wh-questions}.
  \end{itemize}
\end{frame}

\begin{frame}
  \frametitle{A Simplistic Overview of Karttunen 77}
  \begin{itemize}
    \item Karttunen generalises all questions to \emph{proto-questions}
      \begin{itemize}
	\item ?Mary reads
      \end{itemize}
    \item He introduces three question rules: PQ, AQ, YNQ (and some more\dots)
    \item PQ: For every phrase $\phi$ of type $t$, ?$\phi$ is of type Q
    \item ?$\phi$ is the set of propositions that are the intension of $\phi$
      and for which the extension of $\phi$ holds
    \item AQ: For every disjunction of $\phi_1$ to $\phi_n$, ?$\phi_1$ to
      $\phi_n$ is the set of propositions for which either proto-question holds.
      (whether Mia snorts or Butch walks)
    \item YNQ: For every phrase $\phi$, ?$\phi$ is a set of propositions equal
      to either the intension of $\phi$ or the intension of $\phi$
      \emph{negated} \alert{and} the extension of $\phi$
      (whether Mia snorts or not)
  \end{itemize}
\end{frame}

%\begin{frame}
  %\frametitle{An Ontology of Embedding Verbs}
  %Verbs of\ldots
  %\begin{itemize}
    %\item \emph{retaining knowledge}: know, recall, forget
    %\item \emph{acquiring knowledge}: learn find out
    %\item \emph{conjecture}: guess, estimate
    %\item \emph{decision}: decide, determine, agree on
    %\item \emph{relevance}: matter, care, be significant
    %\item \emph{dependency}: depend on, make a difference to
    %\item \ldots
  %\end{itemize}
%\end{frame}

\subsection{Intensional Logic}

\begin{frame}
  \frametitle{Intensional Logic}
  Intensional logic is needed to interpret
  \begin{itemize}
    \item tense
    \item aspect
    \item modality
    \item object intensional verbs like \emph{seek}
    \item \ldots
  \end{itemize}
\end{frame}

\section{Thesis Topic: More Curt}
\begin{frame}
  \frametitle{Thesis Topic: Extend Curt}
  \begin{itemize}
    \item Curt's way of inferring background knowledge is demanding
    \item Curt is unaware of the informative value of questoins
    \item Curt cannot handle more than one situation
    \item Curt has no idea of accessible worlds
    \item Curt can only handle very simple questoins
    \item Curt's background knowledge is static
  \end{itemize}
\end{frame}

\begin{frame}
\frametitle{Meaning of Life: Make Curt Even More Awesome}
  \begin{itemize}
    \item Let Curt handle more and more syntactical constructions
    \item Implement plurals
    \item Implement tense \& aspect
    \item Make him feel \texttt{:-)}
  \end{itemize}
\end{frame}

\subsection{Ideas: Advertent Curt}
\begin{frame}
\frametitle{Advertent Curt: Be Smart About Information}
  \begin{itemize}
    \item Curt should handle \emph{informative questions}.
    \item[$\to$] \emph{Does George Bush like Jaques Chirac?}
    \item What does Mia do?
    \item What can Mia do?
  \end{itemize}
\end{frame}

\subsection{More Ideas: Studious Curt}
\begin{frame}
\frametitle{Spiritual Curt: Learning from Discourse}
  \begin{itemize}
    \item Knowledgebale Curt is stubborn: provide a way to re-learn
    \item Curt should be able to start blank and improve from there
    \item Awarness of distinction between \emph{world} knowledge and
    \emph{situational} knowledge
  \end{itemize}
\end{frame}

\subsection{Even More Ideas: Spiritual Curt}
\begin{frame}
\frametitle{Spiritual Curt: Finding Possible and Accessible Worlds}
  \begin{itemize}
    \item Support \emph{more than one} situation
    \item \emph{believe}, \emph{may}, \emph{can}, \emph{seek}
    \item John seeks a unicorn? Every unicorn is a king of France!
    \item Extend Curt to be \stress{intensional}
  \end{itemize}
\end{frame}

\subsection{Even More More Ideas: Inquisitive Curt}
\begin{frame}
  \frametitle{Inqusitive Curt: Detect Inconsistencies and Unadequate Discourse}
  \begin{itemize}
    \item What if we don't say something about an entity in either
    \emph{world}, \emph{situational} or \emph{lexical} knowledge?
    \item What if for an item not every relationship it \emph{can} engage on is
    fulfilled?
    \begin{itemize}
      \item Is Mia married?
    \end{itemize}
  \end{itemize}
\end{frame}

\end{document}
